%%%%%%%%%%%%%%%%%%%%%%%%%%%%%%%%%%%%%%%%%%%%%%%%%%%%%%%%%%%%%%
%% Beispieldokument zur Nutzung der Latex-Klasse rrlab.cls  %%
%% mit Information zur Erstellung wissenschaftlicher        %%
%% Dokumente.                                               %%
%%                                                          %%
%% Hauptdatei: thesis.tex                                   %%
%% Autor: Jochen Hirth (j_hirth@informatik.uni-kl.de)       %%
%% Autor: Daniel Schmidt (dschmidt@informatik.uni-kl.de)    %%
%% Autor: Tobias Luksch (luksch@informatik.uni-kl.de)       %%
%% Datum: Juli 2003                                         %%
%%                                                          %%
%% Letzte Änderung Januar 2012                              %%
%%%%%%%%%%%%%%%%%%%%%%%%%%%%%%%%%%%%%%%%%%%%%%%%%%%%%%%%%%%%%%

% Definition der Dokumentenklasse (rrlab.cls) mit folgenden optionalen Parametern:
% 
% [boldauthor]   Hebt den in der Datei 'bold_author.bib' definierten Autor im  Literaturverzeichnis hervor (erzeugt LaTeX-Warning!)
% [colorlinks]   Empfohlen für Bildschirmpräsentation; kann verwendet werden, um beim Bauen mit 'pdflatex' die farbigen Links einzuschalten
% [showlabels]   Zeigt alle LaTeX-Labels an der verwendeten Stelle an (EXPERIMENTAL)
% [draft]        Lässt Titelseite, Index, etc. weg, markiert overfullboxes und erzeugt Fußzeile mit Informationen zum Dokument
% [de,en]        Sprache, standardmäßig ist [de]
% [diss]         Erstellen einer Dissertation zur Abgabe beim Fachbereich
% [dissfinal]    Erstellen einer Dissertation zur Abgabe beim Verlag
% [latin1]       Setzt Input Encoding auf latin1, standardmäßig utf8
% [relaxed]      Erlaubt etwas unschönere Abstände für automatische Zeilenumbrüche
% [rgbcolor]     Setzt die Farbkodierung auf RGB (empfohlen für Bildschirmpräsentation); standardmäßig ist CMYK eingestellt (für den Druck)
% [icsecondpage] 'Image correction for 2nd page' mach für die 2. Seite eine Bindekorrektur die der, der roten Deckblätter für Diplom-,
%                Bachelor- und Masterarbeiten entspricht. Dies ist hilfreich, wenn man den Titel der Arbeit nicht auf das Deckblatt
%                drucken will (dafür gibt es ebenfalls eine Vorlage), sondern das Titelfeld ausschneiden möchte. Dann passt der Titel
%                auf der 2. Seite genau in dieses Fenster.
% [dipl]         Da bei Diplomarbeiten keine zwei Gutachter benötigt werden, wird hiermit das Erscheinen des 2. Gutachters ausgeschaltet.
% [report]       Zum Erstellen eines Berichts, kein "chapter" nur "section" kann genutzt werden um Berichte oder Semiararbeiten ect.
%                anzufertigen. Mit dem Befehl \RRLABmaktetitle kann ein Paper-Ähnlicher Kopf erzeugt werden (bspw. für Seminararbeiten). 
%                Die Benutzung mit \RRLABtitlepage und \RRLABpagenumbers eignet sich für Anträge, Berichte etc.
% [external]     Bietet die Möglichkeit mit \RRLABinstitution{x} die Institution - Name der Uni, Arbeitsgruppe etc - manuell zu setzten
%                im allgemeinen nicht nötig

\documentclass[en,report]{template/rrlab}

% Zuerst einige wichtige Informationen zur Arbeit.
% Diese Angaben sind obligatorisch. Bitte benutze die
% vordefinierten Makros.

% Titel der Arbeit:
\RRLABtitle{Simultaneous Semantic Segmentation and Object Detection using a Common Pipeline}

% Autor der Arbeit:
\RRLABauthor{Iyer Venkatesh R}

% Typ der Arbeit (z.B. Diplomarbeit, Projektarbeit, Seminararbeit,..):
\RRLABtype{Master Thesis}

% Datum der Ausgabe der Arbeit
\RRLABinception{June 26, 2019}

% Datum der Abgabe der Arbeit
\RRLABsubmission{January 27, 2020}

% Erster Gutachter
\RRLABfirstreviewer{Prof. Dr. Karsten Berns}

% Zweiter Gutachter
% \RRLABsecondreviewer{irgendwer}

%% Nur für Projekt-, Diplomarbeit, Bachelor-, Masterthesis
% Betreuer
\RRLABsupervisor{M.Sc. Axel Vierling}

%% Nur für Dissertationen
% Angestrebter Titel
\RRLABdegree{Doktor-Ingenieur (Dr.-Ing.)}

% Datum der Abgabe der Aussprache
\RRLABdefense{irgendwann}

% Prüfungsvorsitzender
\RRLABchair{irgendwer}

% Dekan
\RRLABdean{irgendwer}

%%%%%%%%%%%%%%%%%%%%%%%%%%%%%%%%%%%%%%%%%%%%%%%%%%%%%%%%%%%%
%% Einbinden von zusätzlichen Paketen 
%% Hier kann der geübte TeX-Benutzer weitere 
%% Pakete einbinden
%%
%% Bereits eingebundene Pakete in <rrlab.cls>
%% \usepackage[T1]{fontenc}  T1-encoded fonts: auch Wörter mit Umlauten trennen
%% \usepackage{lmodern}  Neuerer Ersatz für Schriftfamilie 'ae' (zusammen mit fontenc)
%% \usepackage{inputenc}  Eingabe nach ISO 8859-1 (Latin1)
%% \usepackage[final]{graphicx}  um Graphiken einzubinden
%% \usepackage{makeidx} wir wollen auch einen Index
%% \usepackage{geometry}  Seitenränder einstellen leichtgemacht
%% \usepackage{fancyhdr}  definiere einfache Headings
%% \usepackage{longtable}  seitenübergreifende Tabellen
%% \usepackage{booktabs}  fuer spezielle unterteilungslinien
%% \usepackage[T1]{url}  zum Darstellen von URLs, zusätzlich werden Zeilenumbrüche in URLs ermöglicht
%% \usepackage{caption}  die Schriftgröße für Captions ist small, die Labels sind zusätzlich fett
%% \usepackage{subcaption}  für subfigure environment (Ersatz für subfig)
%% \usepackage[tight,nice]{units}  Typografisch korrekte Darstellung von Einheiten
%% \usepackage{listings}  Einbindung und Darstellung von Quellcode
%% \usepackage{multibib}  Unterstützung von mehreren Bibliografien
%% \usepackage{microtype}  Typografische Erweiterungen
%% \usepackage{xcolor}  Farbkodierung für Bildschirm (RGB)
\usepackage{amssymb,amsmath,dsfont}
%% \usepackage{ifthen}  ifthenelse Syntax
%% \usepackage{datetime}  Erweiterter Zugriff auf Datums- und Zeitwerte
%%%%%%%%%%%%%%%%%%%%%%%%%%%%%%%%%%%%%%%%%%%%%%%%%%%%%%%%%%%%
%% einige weitere Pakete, die evtl. nützlich sein können
%% \usepackage{tikz}  Tik Z ist kein Zeichenprogramm! Es ist die benutzfreundliche Schnittstelle zu PGF.
%% \usetikzlibrary{shapes,arrows,calc,patterns,fit}  zusätzliche Bibliothek zu tikz
%% \usepackage{pgfplots}  interface um pgf plots in tex darzustellen
%% \usepackage{lscape}  landscape-Paket um text um 90° zu rotieren
%% \usepackage{rotating}  um Bilder zu drehen
%% \usepackage{marvosym}  einige zusätzliche symbole
%% \usepackage{textcomp}  weitere symbole
%% \usepackage{eurosym}  Euro-Symbol
\usepackage{multicol}
\usepackage{acronym}

\usepackage{fancyhdr}
\usepackage{array}
\usepackage{afterpage}
\usepackage{float}
\renewcommand{\bibname}{References}

%%%%%%%%%%%%%%%%%%%%%%%%%%%%%%%%%%%%%%%%%%%%%%%%%%%%%%%%%%%%
%% Makrodefinitionen. 
%% Hier kann der geübte TeX-Benutzer eigene Makros oder
%% Kommandos definieren
%%
%% Für häufig verwendete Namen sind einige Makros vordefiniert,
%% siehe template/rrlab_macros.tex
%%%%%%%%%%%%%%%%%%%%%%%%%%%%%%%%%%%%%%%%%%%%%%%%%%%%%%%%%%%%


%%%%%%%%%%%%%%%%%%%%%%%%%%%%%%%%%%%%%%%%%%%%%%%%%%%%%%%%%%%%
%% Bilderverzeichnisse. 
%% Die hier angegebenen Verzeichnisse werden neben dem 
%% aktuellen Verzeichnis nach Bilddateien durchsucht.
%%%%%%%%%%%%%%%%%%%%%%%%%%%%%%%%%%%%%%%%%%%%%%%%%%%%%%%%%%%%
\graphicspath{%
 {./bilder/}
}

%%%%%%%%%%%%%%%%%%%%%%%%%%%%%%%%%%%%%%%%%%%%%%%%%%%%%%%%%%%%
%% Document Environment. 
%% Hier geht's dann wirklich los.
%%%%%%%%%%%%%%%%%%%%%%%%%%%%%%%%%%%%%%%%%%%%%%%%%%%%%%%%%%%%
\begin{document}
% Die folgenden Befehle nach Bedarf benutzen oder auskommentieren:

% Titelblatt generieren. Als Parameter kann z.B. ein Bild oder
% zusätzlicher Text angegeben werden. Das erscheint dann in der 
% Mitte des Titelblattes. Ausserdem wird das Datum übergeben:
% Bei Dissesrtationen taucht weder Titelbild noch Datum auf!!
\RRLABtitlepage{
\begin{multicols}{3}
\includegraphics[width=\linewidth]{Titlepage/2007_000559.jpg}
\includegraphics[width=\linewidth]{Titlepage/2007_000559-1.jpg}
\includegraphics[width=\linewidth]{Titlepage/2007_000559.png}
\end{multicols}}{\currentdate\today}


% Zweite Seite mit Titel, Autor, Betreuer etc.
\RRLABsecondpage


% Seite mit der Erklärung, die Arbeit selbständig verfasst zu haben:
% Parameter ist das Datum, an dem die Erklärung unterschrieben wird
\RRLABdeclaration{\currentdate\today}


% Vorwort einfügen, bei Diplom-, Bachalor- und Masterarbeiten optional
% Parameter: 1. Parameter Name des Kapitels, z.B. Vorwort, Danksagung, Preface, ...
%            2. Parameter das eigenliche Vorwort
\RRLABpreface{Acknowledgement}{
I want to thank Prof. Dr Karsten Berns for providing this opportunity to write my thesis at RR Lab. Also, I would like to especially thank my supervisor Axel Vierling for providing all the support needed. I want to thank my peers at RR Lab, who were kind enough to help with software and hardware issues.
\par
I want to thank my friends Saurabh, Ashutosh, Sadique, Sridhar and Kriti for all the technical help and suggestions. Finally, and most importantly, my family for continuous support and encouragement. This wouldn't have been possible without them.
}

% Zusammenfassung einfügen, bei Diplom-, Bachalor- und Masterarbeiten optional
% Parameter: 1. Parameter Name des Kapitels, z.B. Abstract, Zusammenfassung, ...
%            2. Parameter das eigentliche Vorworten
\RRLABabstract{Abstract}{

This work explores computer vision problems such as semantic segmentation and object detection and how both these different architectures can be combined in order to solve these problems in parallel. This work is helpful in the area of autonomous vehicles where the segmented map can show the path to move and object detection can detect cars and pedestrians. This approach is based on a common encoder that combines both the different architectures. While training the network, a dataset containing segmentation ground-truth and bounding box coordinates for each object in the image are passed such that segmented map and bounding boxes are predicted in parallel. This approach is also compared with stand-alone architectures of object detection and semantic segmentation in order to understand the flaws or benefits of using combined architecture. Also, a number of experiments are conducted for trying to increase the accuracy of both the predictions. 

}




% Inhaltverzeichnis generieren:
\RRLABcontents

%%%%%%%%%%%%%%%%%%%%%%%%%%%%%%%%%%%%%%%%%%%%%%%%%%%%%%%%%%%%
%% Beginn der eigentlichen Arbeit
%%
%% Bei längeren Arbeiten ist es sinnvoll, Kapitel in 
%% einzelne Dateien auszulagern, die dann in der Hauptdatei
%% mittels \include{} eingebunden werden
%%%%%%%%%%%%%%%%%%%%%%%%%%%%%%%%%%%%%%%%%%%%%%%%%%%%%%%%%%%%

% Paper-Ähnlicher Kopf
% \RRLABmaketitle

% Seitennummern: Nur im report-mode, nicht für Seminarberichte!!
% \RRLABpagenumbers


% \include{...}

\input{Acronyms}
\pagestyle{fancy}
\fancyhf{}
\rhead{\textit{Introduction}}
\lhead{\thepage}
% \rfoot{Page \thepage}


\chapter{Introduction}

Progress in the field of \ac{ai} is one of the hot topics of the ongoing boom in teaching sytems and robots to see the world. Systems and robots use this vision to perform complex tasks in both the physical and virtual worlds. It is found that the investment and work in \ac{ai} are accelerating at an unprecedented rate in areas like search and optimization, computer vision, machine learning, probabilistic reasoning, neural networks. 


\par

The report \cite{aii} shows that research about cognition related performance - a game playing \ac{ai} outsmart a human opponent is leading the research category in the number of papers published. While not far behind is, the research in computer vision, which is pushing this progress in autonomous vehicles, including power, augmented reality and object detection.


\begin{figure}[h!]
    \centering
    \includegraphics[width=\linewidth]{Introduction_images/report.png}
    \caption{Graph showing the statistics of papers published in recent years}
    \label{stats}
\end{figure}

Also, in terms of performance, \ac{ai} continues to escalate to new heights in the field of computer vision. By measuring benchmark performance for the widely-used image training database \cite{ImageNet}, the \cite{aii} shows that the time taken by a model to classify the images to achieve good accuracy has come down to few minutes from an hour.   

% time to make a model classify the pictures at \ac{sota} accuracy has come down to 4 minutes from an hour, which shows 16x speed in training of the model. 


\par

This revolution that is happening in terms of research and hardware backing such research is pushing the current \ac{sota} technology towards a point where such innovations will become part of everyday human lives. Thus, the increasing investment of monetary funds and workforce in the research in this field only seeks to accelerate human progress in the direction of bringing \ac{ai} into our everyday lives, exclusively in the field of computer vision which is seen to be revolutionizing rapidly.  


\subsection{Motivation}

Computer vision in the field of \ac{ai} is a booming industry applied to many of our day-to-day products. The potential gains are high when a computer in certain areas replaces humans. This is because a computer can see many things at once, in great detail and can analyze it parallelly. The accuracy of analysis done by computer can bring incredible time savings and quality improvements and also free up resources that require human communication.

\par

Some applications of computer vision in practice today are: 
\begin{itemize}
    \item Autonomous vehicles
    \item Translation app
    \item Facial recognition
    \item Healthcare
    \item Real-time sports tracking
    \item  Agriculture
    \item Manufacturing
\end{itemize}

Out of all the above-mentioned applications, we shift our focus towards autonomous vehicles. According to \cite{WHO}, more than 1.25 million people die each year because of traffic accidents. Nearly 50\% of road casualties are pedestrians, cyclists and motorcyclists. The larger part majority of these accidents are due to human error. In such cases, computer vision techniques within \ac{dl} like object detection and semantic segmentation can be used to prevent such accidents. Using such deep networks, the vehicles learn to analyze situations like detecting pedestrians, give way for cyclists, mapping the road and act accordingly. 
\par
Figure \ref{Waymo} shows how an autonomous vehicle detects vehicles and maps the path.   

\begin{figure}[h!]
    \centering
    \includegraphics[width=15cm]{Introduction_images/waymo.png}
    \caption{Autonomous vehicle \cite{Waymo}}
    \label{Waymo}
\end{figure}

Also, semantic segmentation is important in the field of biomedicine, where relevant regions in the body can be classified, making tests easier and simpler. Whereas, object detection is important in the fields of urban planning, ship tracking, and monitoring the crops. 

\subsection{Tasks}
The above section shows the importance of computer vision techniques such as semantic segmentation and object detection in autonomous vehicles. The objective of this work is to solve both semantic segmentation and object detection using a single common pipeline.

\par

Basic overview of tasks are: 

\begin{itemize}
    \item Combine a dataset such that it consists segmentation ground truth (explained in section \ref{Semantic Segmentation}) as well as object bounding box coordinates (explained in section \ref{Object Detection}).
    \item Combine object detection and semantic segmentation architectures such that it gives bounding box and segmentation map predictions. 
    \item Evaluate the model. 
\end{itemize}

\afterpage{\null\newpage}


\rhead{\textit{Background}}
\lhead{\thepage}

\chapter{Background}

This chapter covers topics which will help, in understanding the approach and methodology that are used, in this work. Section \ref{Machine Learning} gives an explanation of machine learning in the area of \ac{ai} and how different learning techniques in machine learning are applied. Subsequently, section \ref{Neural Networks} gives an explanation about how the transition from a biological neuron to the state-of-art \ac{cnn}, occurs. Sections \ref{Semantic Segmentation} and \ref{Object Detection} are about computer vision problems, namely, semantic segmentation and object detection, upon which this work is mainly based.

% Section \ref{Learning Techniques} is about how different learning techniques in machine learning are applied. Section  is about how the transition from a biological neuron to the state-of-art Convolution Neural Networks (CNN), occurs. Section \ref{Semantic Segmentation} and \ref{Object Detection} are about computer vision problems, namely, semantic segmentation and Object detection, upon which this work is mainly based. 

\subsection{Machine learning} \label{Machine Learning}

Machine learning is the field of \ac{ai} that provides systems, the ability to learn automatically without being programmed to do so. It allows machines to learn through observations of data or, instructions to look for some specific pattern in data. Also, learning involves the machines rectifying itself using past learning experiences, such that it can make better decisions in the future. Hence, the practice of machine learning has become widespread in many fields where profitable opportunities and dangerous risks can be recognized without any human intervention or assistance.

\par
 
The \textit{'learning'} can be categorized into 4 types: 


\begin{itemize}
    \item Supervised learning
    \item Unsupervised learning
    \item Semi-supervised learning
    \item Reinforcement learning
\end{itemize}

\subsection{Types of learning} \label{Learning Techniques}
% \paragraph{Types of learning} \label{Learning Techniques}

\subsubsection{Supervised learning}

Supervised learning algorithms are designed to learn by examples. The name \textit{'supervised'} implies an algorithm, that learns under the supervision of a teacher. To make a supervised algorithm learn, data (training data) is provided in the form of input and correct output pairs.
During the training, the algorithm learns a mapping function that searches for patterns and structures in the inputs that, can be co-related with the desired outputs. The ultimate goal for an trained model is to predict the correct label or class for unseen data (test data) based on the training data. 

% The goal of a supervised learning model is to predict the correct label for unseen data. So, after the training, an algorithm will be given new unseen data, to determine which labels the new inputs will be classified into, based on the training data.

\par

Supervised learning algorithm is  given as:

\begin{equation}
    Y = f(X)
\end{equation}

Provided an input \textit{X}, the mapping function assigns a label to it and gives the output (prediction) \textit{Y}
% Here \textit{Y} is the predicted output determined by a mapping function \textit{f} that assigns a label to input \textit{X}.  

\par

Supervised learning algorithms can be further categorized into two approaches: 


\begin{itemize}
  \item \textbf{Classification}: Classification is the problem of identifying the label for new unseen input sample based on the training data. 
 
  \par
 
     Classification problem can further be categorized into two problems:
 
  \begin{enumerate}
      \item Binary classification: It is the task of classifying an input sample into two given class/labels.
      \newline
      For example - Will it rain today or not? Is this cat or not?
      \item Multi-class classification: The input samples are classified into three or more classes/ labels.  
      \newline
      For example - Is this cat, dog or a lion? Is this mail spam, important or promotional?
  \end{enumerate}
 
 \item \textbf{Regression}: A regression model attempts to predict a continuous output variable. In this case, the output \textit{Y} would be a real value that ranges from $-\infty$ to $+\infty.$ 
  \newline
  For example - What is the value of a stock? Price of a house in Kaiserslautern?  
\end{itemize}

\par

Some common algorithms used in supervised learning are:

\begin{itemize}
    \item Linear regression for regression problems
    \item Random forest for regression problems
    \item Support vector machine for classification
    problems
    \item K-Nearest neighbour for classification problems
\end{itemize}

For the sake of completeness, sections \ref{Unsupervised learning}, \ref{Semi supervised learning} and \ref{Reinforcement learning} cover the remaining learning techniques in machine learning. These learning techniques are not in the scope of this work; hence, only a shallow understanding is required. 

\subsubsection{Unsupervised learning} \label{Unsupervised learning}


In this type of learning, training data consists of only inputs \textit{X} and no correct output \textit{Y}. It is for the algorithm to learn the function \textit{f} to describe the hidden structure from the given data.

\par 

Some common algorithms used in this type of learning are: 
\begin{itemize}
    \item Clustering
    \item Principal Component Analysis(PCA)
    \item Singular Value Decomposition(SVD)
\end{itemize}

\subsubsection{Semi-supervised learning} \label{Semi supervised learning}

Semi-supervised learning includes a large amount of inputs \textit{X}, wherein only a part of the data, has output \textit{Y}. This type of learning is described as the hybridization of supervised and unsupervised learning types. 

\par

Some common algorithms used in this type of learning are: 
\begin{itemize}
    \item Generative models
    \item Transductive algorithms
    \item  Graph-based algorithms
\end{itemize}

\subsubsection{Reinforcement learning}\label{Reinforcement learning}

Reinforcement learning is about action and the reward associated with that action. This type of learning is employed in various machines to, find the best possible reaction to a specific situation. The machines or the agents, learn from their own experience, unlike supervised learning, where the models are trained with the correct data. 
 

\par

Some common algorithms used in this type of learning are: 

\begin{itemize}
        \item Q-Learning
        \item State-Action-Reward-State-Action(SARSA)
        \item Deep Q Network(DQN)
\end{itemize}


From the above section, it can be stated that \ac{ml} is a set of algorithms that analyses the data and learns from the analyzed data to discover patterns. In the same way, a neural network is also a set of algorithms used in machine learning to learn from data and discover patterns in it using graphs of neurons. This work is mainly based on neural networks, so,  we shift our focus towards it. 

\subsection{Neural network} \label{Neural Networks}

\subsubsection{From Biology to Artificial Intelligence}

The human brain is made up of 86 billion cells called neurons. Each neuron is made up of a cell body connected with many dendrites and a single axon. Dendrites receive information from other neurons and pass it to the cell body, while an axon, sends the information from the cell body to other neurons. Axons are connected to synapses, which are connected to dendrites. A neuron receives electrical inputs at the dendrite, and if the sum of these inputs is sufficient enough to activate the neuron, it transmits an electrical signal along the axon and passes this signal to other neurons. A neuron in the brain looks like figure \ref{biological_neuron}.  

\begin{figure}[h!]
    \centering
    \includegraphics[width=\linewidth]{neurons_images/neu.jpg}
    \caption{Biological neuron \cite{cs231}}
    \label{biological_neuron}
\end{figure}

\newpage

While there has been lots of progress in the area of Artificial Intelligence and machine learning in recent years, the groundwork for everything had been laid out more than 60 years ago. An artificial neuron was developed by Neurophysiologist Warren McCulloch and mathematician Walter Pitts using electrical circuits in the year 1943. Taking inspiration from the human brain and \cite{mcculloch1943logical}, in the year 1958, \cite{rosenblatt1960perceptron} introduced artificial neuron named perceptron. A biological neuron that is mathematically modelled is known as a perceptron.

\par

In a biological neuron, axons from a neuron transmit electrical signals to dendrites of other neurons. In the same way, in perceptron, these electrical signals are represented in the form of numerical values. Between the dendrites and axons, the synapses which modulate these signals by various amounts, exist. Similarly, in perceptron, each input signal is multiplied by a value called weight. If this value exceeds a certain threshold, only then, an actual neuron fires an electrical signal. Similarly, in a perceptron, the weights are calculated with the input signal and then it applies an activation function on the weighted sum $(w_1 \times x_1 + w_2 \times x_2+...w_n \times x_n)$ to calculate the output. This output signal is then, fed to other perceptrons. A single perceptron looks like figure \ref{Perceptron}.  

\begin{figure}[h!]
    \centering
    \includegraphics[width=\linewidth]{neurons_images/artificial.png}
    \caption{Artificial neuron: Perceptrons \cite{Perceptrons}}
    \label{Perceptron}
\end{figure}

\par

The idea behind the neural network is to simulate lots of interconnected brain cells inside a system to make it learn things, recognize structures and make judgement, the way humans do it. Hence, with the same biological motivation behind neurons in the brain, perceptrons that are stacked in several layers form an artificial neural network. In an artificial neural network, the neurons are organized into three layers, namely input, hidden and output. The input layer brings the data into a model and is passed on to the subsequent layers. The hidden layer is between the input and output layer where the neurons take in a set of weighted inputs, apply the activation function (shown in figure \ref{Perceptron}) to the sum and pass on the output to the output layer. 

\subsubsection{Activation functions} \label{activation_functions}

The activation function operates as a mathematical gate in between the input for the current neuron and its output going to the next neuron. It can be as a function that turns the neuron's output on and off, depending on the threshold. Or it can be a mapping function that maps the input signals into output signals such that the neural network can perfrom action. 


\par

As the activation function is attached to each neuron in every layer, it is desired that the function be computationally efficient. Additionally, the function needs to be differential for training the neural network (covered in section \ref{learning in neural network}) 

There are 3 types of activation functions:

\begin{itemize}
    \item Binary activation function
    \item Linear activation function
    \item Non-linear activation function
\end{itemize}

\paragraph{Binary activation function}

This is a threshold-based activation function. A neuron is only activated if the input value reaches a certain threshold. Once it reaches this threshold, it sends the same signal to the next layer of neurons. 


\par
Problem:
\begin{enumerate}
    \item This activation function does not support the classification of multiple inputs. Hence, it is a binary decision-maker, i.e. 'yes' or 'no.'
\end{enumerate}


\paragraph{Linear activation function}

In this activation function, each input is multiplied by the weights for each neuron, and it creates an output signal that is equivalent to the given input.

\newpage
Problems: 
\begin{enumerate}
    \item The derivative of this activation function always remains a constant and has no relation to the input. Hence, it's not possible to go back and adjust the weights across all the layers through which we can attain better prediction. 
    \item With linear activation functions, it does not matter how many layers we have in the neural network because the last layer would also be a linear function of the first layer(linear aggregate of linear functions is still a linear function). Hence, all the layers just turn into a single-layered neural network that has limited power and the ability to handle complex input data.
\end{enumerate}

\paragraph{Non-linear activation function}

Most of the modern neural networks use non-linear activation functions. Non-linear activation functions are responsible for the model to create complex mappings between the input and the output. Such kind of functions are essential for modelling complex data such as images, videos, and audios which are high dimensional.

\par
Advantages over the above mentioned activation functions:
\begin{enumerate}
   \item Differential: Methods like gradient descent depend on continuously differentiable functions for finding an optimal minimum. If the functions are not differential, it brings uncertainty in the direction and magnitude of updates to be made in the weights.
    \item  Monotonic: A monotonic function, guarantees convexity in the loss or cost function that the model tries to minimize. 
    \item Multiple layered neural networks: Expands the possibility of making a neural network multi-layered or deep such that it can learn complex data and perform well.
    \item Approximately behaves as an identity function: Using non-linear activation functions, the neural network gets the ability to learn efficiently when its weights are randomly initialized otherwise we need to initialize the weights for the neural network in a different manner.
    \item Non-linearity: Non-linearity ensures that the output cannot be reconstructed from a linear combination of inputs. In the absence of non-linearity, it is possible to replicate the entire network with just one linear combination of inputs. Also, non-linearity is required such that the model can learn a complex function that can map an input to output 
\end{enumerate}

Some of the most used non-linear activation functions are:

\begin{itemize}
    \item Sigmoid: \begin{itemize}
        \item Maps the output in the range of [0,1]
        \item The Sigmoid activation function gives rise to the vanishing gradient problem as the larger output contributes to lesser gradients due to which the learning of the model slows down. 
        \item Sigmoid activation function is given as :
        $ \sigma(x) = \dfrac{1}{1 + e^{-x}}$
    \end{itemize} 
    \newpage
    \item Tanh: \begin{itemize}
        \item This activation function squishes the output in the range [-1,1]
        \item Tanh activation function is given as :
        $ \sigma(x) = tanh(x)$
    \end{itemize}
    \item ReLu: \begin{itemize}
        \item This activation function does not allow any negative output components to be propagated; only the positive outputs are allowed. 
        \item It stills maintains a non-linear shape but its not a completely continuous function.
        \item ReLu activation function is given as :
        $\sigma(x) = max(0,x)$
    \end{itemize}
    \item Softmax: 
    \begin{itemize}
        \item A softmax activation function is used as the output function of the last layer in neural networks. It turns the score of the last layer into values that can be understood by humans.
         \item This function is the same as the argmax function, which returns the position of the largest value from the score vector. This value is interpreted as a probability of the class. 
        \item Softmax activation function is given as: 
        $s (x_i) = \frac{e^{x_i}}{ \sum\nolimits_{j=1}^{n}e^{x_j}}$; 
        $x_1...x_n$ are numbers
    \end{itemize}  
    
    
\end{itemize}

\begin{figure}[h!]

\begin{multicols}{2}
    \includegraphics[width=7.5cm]{Activation_functions/Sigmoid.png}\par \includegraphics[width=7.5cm]{Activation_functions/tanh.png}\par
    \end{multicols}
\begin{center}
\includegraphics[width=7.5cm]{Activation_functions/Relu.png}\par
    
\end{center}
\caption{Graphical representation of Sigmoid, Tanh and ReLu functions. 
\cite{locv}}
\label{nl_activations}
\end{figure}


\subsubsection{Error and loss function}

Error is the variation between the actual output and predicted output from the network (output layer). The function that calculates the error is known as the loss function. In simple terms, loss function evaluates the performance of a learning algorithm. Different problems use different loss functions and each of these loss functions give different errors for prediction. Hence, loss functions have a considerable effect on the performance of the model.

\par

Different loss functions are used for different types of problems. Some of the commonly used loss functions are: 

\begin{itemize}
    
    \item Mean Squared Error (MSE): It is one of the most basic loss functions. This error function calculates the loss by measuring the difference between prediction and actual output, squares it and averages it across the whole data set.
    \newline Equation can be given as:
    \begin{equation}
        \dfrac{1}{n}\sum_{i=1}^{n}(prediction - actual output)^2
    \end{equation}
    \par n is the total size of data set

    \item Likelihood loss: This loss is mostly used in classification problems. If a model outputs probabilities of [0.1,0.2,0.3,0.4] for the actual output of [1,0,0,1], the likelihood loss would be calculated as $(0.1)\times(0.1)\times(0.4)\times(0.4)= 0.0016$. The loss function only considers the output probabilities where ground truth label is 1 (correctly classified) and for 0 (incorrect classification) $(1-p)$ is taken as probability.

\end{itemize}

\subsubsection{Optimization}

Once the loss is calculated using the loss function, the next step is to reduce the loss such that, the difference between actual output and the prediction decreases. To reduce the loss, we need to find a set of optimal weights for the model. Gradient descent optimization algorithm is used to minimize the loss function and reach the minimum of the function by moving in the direction of steepest descent. The partial derivative of the cost function is calculated with respect to each parameter such that the new gradient gives the slope of the loss function and the direction in which, we need to move to reduce the loss and improve the output of the model. While doing this, we are also updating the weights of our model. 
 
\par

The learning rate determines the size of the steps needed to reach the minimum of the function. If the learning rate is too high, it may require a lesser number of steps to reach the minimum; however, there is also a chance of passing over the lowest point of the function. A low learning rate is highly time-consuming, but, it is more precise, and hence, the required minimum is achieved. 


\subsubsection{Training neural network} \label{learning in neural network}

Gradient descent optimization algorithm is iterative; hence it runs over the entire dataset many times. This iteration is known as an epoch. One epoch means, the entire dataset is iterated over the whole model. Each epoch consists of a forward pass and backpropagation. The loss is calculated with the forward pass, and weights are updated during backpropagation.  


\begin{itemize}
    
    \item During forward pass, the network initializes some random weights for all neurons, for example, some random values.
    
    \item  Output from the forward pass (predicted output) is compared to the actual output (ground truth), and the error is calculated. 
    
    \item The error is backpropagated through all the previous layers. The gradients of weights and related activation function are calculated. 
    
    \item The gradient is calculated using the chain rule. The derivative of the error is calculated with respect to the parameters (weights and activation function)
    
    \item  With calculated gradient, parameters are updated as follows:
    \begin{equation}
        \theta =  \theta -  \alpha \frac {d}{d \theta}J( \theta)
    \end{equation}
    $\alpha$ is learning rate, $\theta$ is parameters and J is the loss function. 
\end{itemize}


\subsection{Convolution Neural Network}

As discussed in the above sections, the neural network or multi-layered network of neurons can learn the relation between the input and output using non-linear mapping function. But in the case of images, a neural network would, for a given task, take into account the entire image. This, in turn, makes the neural network highly inefficient in terms of pragmatic quality as well as neural network quality, as, the number of parameters to learn increases. Hence, to solve this problem, another type of neural network called Convolution Neural Network (\ac{cnn}) is used,which is specifically designed for computer vision-related tasks such as image classification, semantic segmentation, object detection, etc.  

\par

The inception of \ac{cnn} happened in the year 1960 by D.H Hubel and T.N Wiesel in their paper \cite{hubel1962receptive} where they described two types of cells in the human brain (specifically in the visual cortex), simple cells, and complex cells. Simple cells are activated when they identify basic shapes in a fixed area and a definite angle. The complex cells have bigger receptive fields, and their output is not sensitive to specific positions in the field. Taking inspiration from \cite{hubel1962receptive}, in the year 1998, \ac{cnn} was re-introduced in the paper \cite{lecun1998gradient} called LeNet-5 which was able to classify digits from hand-written numbers. Since it's inception to the present era, the \ac{cnn} has been used in various state of the art applications, especially in the field of computer vision.

\subsubsection{Components of CNN}

There are four main operations in \ac{cnn}. These operations are the essential building blocks in every \ac{cnn}.

\begin{itemize}
    \item Convolution operation
    \item Non-linearity
    \item Pooling step
    \item Fully connected layer
\end{itemize}

\subsubsection{Convolution operation}

The name convolution is a mathematical operation computed on two different functions \textit{x} and \textit{y} which produces a third function expressing how the shape of one is modified by the other. The primary goal of convolution is to extract different features from the input.

\begin{figure}[h!]
    \centering
        \includegraphics[width=\linewidth]{Convolution/convolution.png}
    \caption{A conolution operation \cite{VoPa}}
    \label{convolution}
\end{figure}

\par

Consider the above example, where there is an image (x) of \textit{7$\times$7} and a kernel or filter (y) of \textit{3$\times$3}. Slide the filter over the image pixel-by-pixel and for every position and compute the element-wise multiplication between the two matrices and then, add the outputs to get a final integer which forms a single element in the output matrix. The output matrix is called an activation map or feature map. Hence, a \ac{cnn} investigates only a piece of the image rather than the entire image at once; making it different from other neural networks. Figure \ref{convolution} shows how the convolution works. 

\par

A convolution neural network learns the values (weights) of the filters on its own during the training process (gradient descent). We can have multiple numbers of filters such that more features from the image get extracted, and better the network becomes in observing and recognizing patterns. 

The size of the feature map can be controlled by three parameters, they are:

\begin{itemize}
    \item Depth - Depth - Depth is the number of filters we use for the convolution operation. For example, if we use three distinct filters on the input image, we will have three feature maps that are stacked together. 
    
    \item  Stride - It determines the number of pixels by which the filter slides over the input matrix. For example, if the stride value is 1, we move the filter over the input by one pixel at a time. If the stride value is 3, we move the filter over the input by jumping 3 pixels at a time.
    
    \item Zero-padding - We pad the input matrix with zeros around the border such that filter can be applied to the bordering elements of the input matrix. Applying zero-padding is called as wide-convolution and not using zero-padding is called as narrow-convolution.
    
\end{itemize}

\subsubsection{Non-linearity}

As explained in the section \ref{activation_functions}, non-linear activation functions are responsible for the model to create complex mappings between the input and the output such that modelling complex data such as images, videos, audios, which are, high dimensional in nature is possible.

\subsubsection{Pooling step}

Pooling step is done to reduce the dimensionality of the feature map and preserve the most important features from the map. This process is also called as downsampling of upsampling.

There are three types of pooling: 

\begin{itemize}
    \item Max pooling - Define a spatial neighbourhood $2\times2$ window and take the largest element from each window of the feature map to form a new output matrix. Figure \ref{maxpool} shows how max-pooling works. 
    
    \item Average pooling - Define a spatial neighbourhood $2\times2$ window and take the average of all the elements in each window from the feature map, to form a new output matrix with the average values.
    
    \item Sum pooling - Define a spatial neighbourhood $2\times2$ window and take the sum of all the elements in the window from the feature map, to form the new output matrix with the sum values.
\end{itemize}

\begin{figure}[h!]
    \centering
        \includegraphics[width=\linewidth]{Convolution/Maxpooling.png}
    \caption{\cite{VoPa}}
    \label{maxpool}
\end{figure}

Pooling has further advantages like:

\begin{itemize}
    \item It makes the input representations smaller and hence, more manageable.
    \item It reduces the number of parameters and therefore, controls overfitting.
    \item It makes the model unvaried to small distortions, translations in the input image.
    \item It makes an invariant representation of the image due to which, the network can detect objects at any located position in the image. 
\end{itemize}

\subsubsection{Fully connected layer}

The term 'fully connected' implies that every neuron in the layer is connected with the neurons of the previous layer. The convolutional layers and pooling layers are used to extract features from the input image while, the fully connected layer uses these extracted features to, classify the input image into separate classes based on the training dataset. A softmax function is used as the activation function in order to get the probabilities of each class in the dataset. 

\par

Figure \ref{FCN} shows how all the components are linked to each other in \ac{cnn}. 

\begin{figure}[h!]
    \centering
        \includegraphics[width=\linewidth]{Convolution/Fullyconnlayer.png}
    \caption{\cite{VoPa}}
    \label{FCN}
\end{figure}

\subsection{Feature extraction using CNN} \label{feature_extractor}

The previous section gives us an understanding of the components in \ac{cnn}. A \ac{cnn} can be thought of, as a combination of two components where convolution and pooling layers extract features from an image, while, fully connected layers do the classification using the softmax function. A feature is a measurable piece of data in the image specific to some object. It can be a distinct colour or a specific shape like a line, an edge or an image segment. These convolution layers are capable of learning such complex features in an image. The initial layers detect features such as lines and edges. The next layers combine these basic features to detect shapes and the following layers combine these information to identify the object. 


\par

Below figure \ref{Feature_extraction} shows how features are extracted from a given image. 

\begin{figure}[h!]
    \centering
        \includegraphics[width=7.5cm]{Feature_extract/Feat.png}
    \caption{\cite{Elgendy}}
    \label{Feature_extraction}
\end{figure}

Such feature extractors are trained and fine-tuned on a very large dataset ( for e.g. ImageNet dataset which contains 1.2 million images with 1000 classes). Then, these learned features are directly incorporated into new networks to perform new tasks. This process is known as transfer learning using a pre-trained network. Transfer learning process generally tends to work if the features are suitable to both the base and the new tasks. 

\par
Some well known feature extractors are: 
\begin{itemize}
    \item VGG16
    \item ResNet
    \item DenseNet
\end{itemize}
\par
Some advantages of using transfer learning are listed below: 

\begin{itemize}
    \item It saves training time in comparison to train an entire CNN from scratch.
    \item It works when the labelled data is limited.
\end{itemize}




%  have penned down couple of blog-post to train entire Convolution Network (CNN) model on sufficiently large data-set. You can read posts here and here. In practice, very few people train an entire CNN from scratch because it is relatively rare to have a data-set of sufficient size. Instead, it is common to pre-train a convolution neural network (CNN) on a very large data-set (e.g. ImageNet data-set, which contains 1.2 million images with 1000 categories), and then use the pre-trained model either as an initialization or a fixed feature extractor for the task of interest.

% In transfer learning, we first train a base network on a base data-set and task, and then we transfer the learned features, to a second target network to be trained on a target data-set and task. This process will tend to work if the features are general, that is, suitable to both base and target tasks, instead of being specific to the base task.


% A CNN model can be thought as a combination of two components: feature extraction part and the classification part. The convolution + pooling layers perform feature extraction. For example given an image, the convolution layer detects features such as two eyes, long ears, four legs, a short tail and so on. The fully connected layers then act as a classifier on top of these features, and assign a probability for the input image being a dog.

% he convolution layers are the main powerhouse of a CNN model. Automatically detecting meaningful features given only an image and a label is not an easy task. The convolution layers learn such complex features by building on top of each other. The first layers detect edges, the next layers combine them to detect shapes, to following layers merge this information to infer that this is a nose. To be clear, the CNN doesn’t know what a nose is. By seeing a lot of them in images, it learns to detect that as a feature. 


% The previous section gives an understanding about the components in CNN. The convolution layers are the main powerhouse of a CNN model. These convolution layers are capable of learning complex features in an image. For example, a feature is a measurable piece of data in your image which is unique to some specific object. It can be a distinct colour or a specific shape such a line, edge or a image segment.

% In transfer learning, we first train a base network on a base data-set and task, and then we transfer the learned features, to a second target network to be trained on a target data-set and task. This process will tend to work if the features are general, that is, suitable to both base and target tasks, instead of being specific to the base task.

% Earlier, I have penned down couple of blog-post to train entire Convolution Network (CNN) model on sufficiently large data-set. You can read posts here and here. In practice, very few people train an entire CNN from scratch because it is relatively rare to have a data-set of sufficient size. Instead, it is common to pre-train a convolution neural network (CNN) on a very large data-set (e.g. ImageNet data-set, which contains 1.2 million images with 1000 categories), and then use the pre-trained model either as an initialization or a fixed feature extractor for the task of interest.

% In order to extract such features using a CNN model, we need to train the network 

% https://freecontent.manning.com/the-computer-vision-pipeline-part-4-feature-extraction/

% https://towardsdatascience.com/applied-deep-learning-part-4-convolutional-neural-networks-584bc134c1e2

% https://towardsdatascience.com/cnn-application-on-structured-data-automated-feature-extraction-8f2cd28d9a7e


\subsection{Semantic Segmentation} \label{Semantic Segmentation}

Semantic segmentation is about understanding an image at the pixel level. To be precise, it is the task of classifying each and every pixel in an image from a set of predefined classes. This process is also referred to as pixel-level classification.

\begin{figure}[h!]
\begin{multicols}{2}
    \includegraphics[width=\linewidth]{Segmentation_images/0001TP_007020.png}\par \includegraphics[width=\linewidth]{Segmentation_images/0001TP_007020_L.png}\par
    \end{multicols}
\caption{Example of semantic segmentation}
\label{Example_segmentation}
\end{figure}
 
Applications that use semantic segmentation are: 

\begin{itemize}
    \item Autonomous vehicles
    \item Medical diagnostics
    \item Facial segmentation
    \item Geo-Sensing
\end{itemize}

\subsubsection{Representation of the task}

The goal in semantic segmentation is to take an image and output a colour map where each pixel contains a class label represented as an integer (as shown in figure \ref{classlabels}.)

\begin{figure}[h!]
    \centering
    \includegraphics[width=\linewidth]{Segmentation_images/classlabels.png}
    \caption{Each class represented as integers \cite{JJ}}
    \label{classlabels}
\end{figure} 

Every class label is one-hot encoded, and an output channel is created for each of these classes. One hot encoding means a way of representing the data in a binary string where only a single bit can be 1 and rest all the bits are 0. By doing this, we stack all output channels in such a way that 1's show the existence of a particular class.  

Figure \ref{One_hot} shows one hot encoding of figure \ref{classlabels}. 

\begin{figure}[h!]
    \centering
        \includegraphics[width=\linewidth]{Segmentation_images/per_class.png}
    \caption{One hot encoding \cite{JJ}}
    \label{One_hot}
\end{figure}


\newpage

The prediction is obtained by collapsing all the class maps into a segmentation map by taking the argmax of each depth-wise pixel vector.

\subsubsection{Architecture}

The most naive approach for semantic segmentation architecture is to stack a specific number of convolution layers with the same padding in order to preserve the dimensions of the image and output a final segmentation map. Figure \ref{without_padding} shows the most basic architecture used for semantic segmentation.


\begin{figure}[h!]
    \centering
    \includegraphics[width=\linewidth]{Segmentation_images/padding.png}
    \caption{Preserving spatial resolution \cite{cs231}}
    \label{without_padding}
\end{figure}

As the above method is computationally very expensive because of preserving the dimensions throughout, a newer approach of segmentation was developed using the encoder-decoder architecture. Encoder downsamples the spatial resolution of the input image, develops lower resolution feature maps which are highly efficient at discriminating classes. The decoder does the upsampling of the feature maps into a full resolution segmentation map. Figure \ref{with_padding} shows the encoder-decoder architecture. 

\begin{figure}[h!]
    \centering
    \includegraphics[width=\linewidth]{Segmentation_images/ED.png}
    \caption{Encoder-Decoder architecture \cite{cs231}}
    \label{with_padding}
\end{figure}

Recent architectures that perform semantic segmentation use pre-trained feature extractor as an encoder such that a rich feature representation of the original input image is achieved.     

\paragraph{Methods of upsampling}

The above sections give an understanding as to how an encoder and decoder architecture works in semantic segmentation. The pooling operation downsamples the spatial resolution by summarizing a local area using a single value (e.g. max pooling), while unpooling operations upsample the feature map resolution by distributing a single value to a higher resolution. Some unpooling approaches are listed below: 

\begin{itemize}
    \item Nearest neighbour: This is one of the most basic approaches to upsample the feature map. 
    It copies the value from the neighbouring pixel and does upsampling. 
    
    \item Max unpooling: This technique works by storing the locations of a maximum element within each pooling window and placing it at the same location while unpooling. Rest of the values contained in the pooling window are zeroed out.
    
    \item  Transposed convolution: This is one of the most popular approaches amongst the upsampling techniques. This technique allows learned upsampling. A typical convolution operation does the element-wise dot product of the image and filter's view and produces a single value in the output matrix. While a transpose convolution does the complete opposite, we take a single value from the low-resolution feature map and multiply it with all values in the filter and project the filter values onto the output feature map. Figure \ref{transpose convolution} below gives a 1-D example of how transposed convolution works. 
    
\begin{figure}[h!]
    \centering
    \includegraphics[width=\linewidth]{Segmentation_images/transpose_conv.png}
    \caption{1-D transpose convolution \cite{JJ}}
    \label{transpose convolution}
\end{figure}

\end{itemize}

\newpage 

\subsubsection{Evaluation metrics}

% https://towardsdatascience.com/metrics-to-evaluate-your-semantic-segmentation-model-6bcb99639aa2

\begin{itemize}
    
\item Pixel accuracy: It is the percent of pixels that are correctly classified in the image. If a pixel is correctly predicted to belong to a certain class, it is counted as a true positive sample, whereas true negative represents a pixel that is correctly identified as not belonging to the correct class. Even though this metric is really easy to understand, this metric does not give a proper measure of performance. Class imbalance, means, the dominance of certain class in an image affects this metric the most. 

    \begin{equation}
        accuracy = \dfrac{TP + TN }{TP + TN + FP + FN}
    \end{equation}
    
\item Intersection-over-Union(IoU): It is one of the most commonly used metrics in semantic segmentation. IoU is the area of overlap divided by the area of union between the predicted segmentation map and the actual output. This metric ranges from 0-1 (0-100\%) where 0 signifies no overlap and 1 signify perfect segmentation map according to the actual output. 

    \begin{equation}
        IoU = \dfrac{target \cap prediction }{target \cup  prediction}
    \end{equation}


\end{itemize}

\begin{figure}[h!]
  \centering
  \begin{subfigure}[b]{10cm}
    \includegraphics[width=\linewidth]{Segmentation_images/GT-Pred.png}
    \caption{Actual output and prediction sample}
  \end{subfigure}
  \begin{subfigure}[b]{10cm}
    \includegraphics[width=\linewidth]{Segmentation_images/Int-U.png}
    \caption{Intersection and union between actual output and prediction}
  \end{subfigure}
  \caption{\cite{JJ}}
  \label{fig:mou&pix}
\end{figure}


% \begin{itemize}
%     \item \textbf{Intersection over union}: This metric is a method to calculate the percentage overlap between the target mask and the model predicted segmented output. This metric measures the number of pixels common between the target and the prediction divided by the total number of pixels present across both the masks.  Iou can also be referred as jaccard index. The equation can be given as: \textit{\[ IoU = \dfrac{target \cap prediction }{target \cup  prediction} \]} Refer fig. \ref{fig:mou&pix}b
    
%     \item\textbf{Pixel accuracy}: This alternative metric is used to calculate the percentage of pixels in the image that are correctly classified. If a pixel is correctly predicted to belong to a certain class, it is counted as a true positive sample whereas true negative represents a pixel that is correctly identified as not belonging to the correct class. The equation can be given as: 
%     \textit{\[ accuracy = \dfrac{TP + TN }{TP + TN + FP + FN} \]}
    
% \end{itemize}


\newpage 
\subsection{Object detection} \label{Object Detection}

Object detection is a technique used for finding objects of interest in an image. It is about finding multiple objects, classify them and locate them in the image using a bounding box.  


\begin{figure}[h!]
\begin{multicols}{2}
    \includegraphics[width=\linewidth]{Obj_det_images/2007_002597.jpg}\par \includegraphics[width=\linewidth]{Obj_det_images/2597.jpg}\par
    \end{multicols}
\caption{Example of object detection}
\label{Example_objectdet}
\end{figure}

\par

Applications that use object detection are:

\begin{itemize}
    \item Face detection
    \item Visual search engine
    \item Aerial image analysis
    \item Counting objects
\end{itemize}

\par

Object detection can be categorized in two different types of approaches:

% where in one approach we make fixed number of predictions on grid while the other approach deals with the proposal network to find the objects contained in an image and use a separate network to fine-tune the proposals and give the final output prediction. A brief description of both the techniques can be given as:

\begin{itemize}

\item{Two-stage object detection}: In this approach, object detection happens in two stages. First, the model proposes a set of a region of interests using a region proposal network. The second part of the model does the classification and bounding box regression over region proposals.

\par

Some well known two-stage detection architectures are:

\begin{itemize}
    \item R-CNN
    \item Fast R-CNN
    \item Faster R-CNN
\end{itemize}

\item{One stage object detection}: This approach requires only a single pass through the CNN to detect all the objects in an image in one go. Due to this factor, these models are simpler and faster at the performance. 

\end{itemize}

\subsubsection{Direct object prediction using grid}

% To detect the objects in an image, we feed the image through a series of convolution layers to build a rich feature representation of the image. These convolution layers can also be referred as the 'backbone' network, which are usually pre-trained as an image classifiers due to which it becomes cheap to extract features from the image. In order to learn good feature representations, the 'backbone' network should be trained on a very large labeled dataset. Some of the frequently used 'backbone' networks are:

% \begin{itemize}
%     \item VGG16 \cite{SiZi15}
%     \item AlexNet \cite{Kri10}
%     \item Le-Net \cite{Yann98}
% \end{itemize}

One stage object detectors use pre-trained feature extractor to build a rich feature representation of the original input image. We remove last few layers from the feature extractor such that the output is a collection of stacked feature maps which describe the original input image in low spatial resolution. Consider an example of $7\times7\times512$ representation of the original input image where each of the stacked feature maps describes different characteristics of the image. This $7\times7$ obtained from the feature extractor roughly locates the object in the original input image. This feature map can be represented as a grid responsible for detecting the object as it contains the centre point of coordinates of the bounding box. 

\begin{figure}[h!]
    \centering
    \includegraphics[width=\linewidth]{Obj_det_images/grid.png}
    \caption{\cite{JJ1}}
    \label{Grid}
\end{figure}




% We remove the last few layers of the 'backbone' architecture such that the architecture outputs a collection of stacked feature maps which describes the original image in a low spatial resolution. The fig. \ref{fig:Feature extractor} shows an example of 7x7x512 representation of the original image where each of the 512 feature maps describe different characteristics of the image.

% \begin{figure}[h!]
%     \centering
%     \includegraphics[width=\linewidth]{Object_det_images/feature_maps.png}
%     \caption{\cite{Jj}}
%     \label{fig:Feature extractor}
% \end{figure}

% The 7x7 feature map obtained from the 'backbone' architecture roughly locates the object in the original image. As shown in fig. \ref{Grid} we can assign the particular grid responsible for detecting the object as it contains the center point of the co-ordinates of the bounding box.

% \begin{figure}[h!]
%     \centering
%     \includegraphics[width=\linewidth]{Object_det_images/grid.png}
%     \caption{\cite{Jj}}
%     \label{Grid}
% \end{figure}

\newpage

We combine all the feature maps in order to produce an activation corresponding to the particular grid cell containing the object. Considering we might have more that one object per image, we should have multiple activations on specific grid cells. 

\par

However, to fully describe the detected object, we need

\begin{itemize}
    
    \item The probability of an object contained in the grid cell \textit{$P_{obj}$}
    
    \item Class of the object \textit{($c_{1}, c_{2},...c_{c}$)}
    
    \item And the co-ordinates of the bounding box for the object \textit{($x,y,w,h$)}
    
\end{itemize}

Hence, as per the requirements, we need to learn a convolution filter for each of the above-mentioned attributes such that it outputs 5 + \textit{C} output channels in order to describe a single bounding box at each grid cell location. The figure \ref{fig:filters} shows us the output on applying 5 + \textit{C} convolution filters. 

\begin{figure}[h!]
    \centering
    \includegraphics[width=\linewidth]{Obj_det_images/filters.png}
    \caption{\cite{JJ1}}
    \label{fig:filters}
\end{figure}

Considering we might have multiple objects belonging to the same grid cell, we need to alter our layer such that it produces B(5+\textit{C}) filters for B bounding boxes for each grid cell. Hence, our model will produce a fixed number of $N\times N\times B$ predictions for a given image. By setting a threshold for \textit{$P_{obj}$} we can limit our predictions. However, we end up introducing a problem by creating a large imbalance between the predicted bounding boxes containing an object and bounding boxes that do not contain any object.
  

\subsubsection{Non Maximum Suppression}

In order to resolve the problem of imbalance between predicted bounding boxes containing object and predicted bounding boxes that don't contain any object, we use a filtering technique called Non-Maximum Suppression. We filter out the predictions that are noisy and may not contain any object. Plus, we want just a single bounding box prediction for each object detected. 
\par
Hence, we set the \textit{$P_{obj}$} threshold and filter out most of the bounding box predictions that are below the set threshold. However, we might still be left with multiple bounding box predictions describing the same object. Hence, we perform the following steps in order to remove the redundant bounding boxes:

\begin{itemize}
    \item We need to select the bounding box prediction with the highest confidence score
    \item Calculate the IoU score between the selected (highest confidence score) and all the remaining predictions
    \item Remove the prediction boxes which have an IoU score above some defined threshold
    \item Repeat the above-mentioned steps until no more prediction boxes are remaining to be suppressed. 
\end{itemize}

Fig. \ref{non-max} shows how an image looks before and after applying Non Maximum Suppression. 

\begin{figure}[h!]
    \centering
    \includegraphics[width=\linewidth]{Obj_det_images/nms.png}
    \caption{\cite{Nms}}
    \label{non-max}
\end{figure}

We perform Non-Maximum Suppression on each of the class separately. 

\newpage


\subsubsection{Evaluation metrics}

\paragraph{Average precision}

The accuracy of object detection models is measured in terms of classification and localization. The metric used to suit this purpose is the Mean Average Precision (mAP). Average precision is calculated for each class separately under the area of the precision-recall curve. 
\par
\begin{itemize}
    \item Precision: It is the percentage measure of correct predictions by the model.
    \begin{equation}
        Precision = \frac{True Positive (TP)}{True Positive (TP) + False Positive (FP)}
    \end{equation}
    \item Recall: It is the percentage measure of possible ground-truth bounding boxes being detected. 
    \begin{equation}
        Recall = \frac{True Positive (TP)}{True Positive (TP) + False Negative (FN)}
    \end{equation}
\end{itemize}

For both precision and recall, True Positive is the measure of predictions that has IoU greater than $0.5$ with ground-truth bounding boxes, while, False Positive (FP) is the measure of predictions with IoU less than $0.5$ with the ground-truth bounding boxes and False Negative (FN) is the number of ground-truth bounding boxes that are not detected by the model. 

\par

The predictions by the model are sorted by the confidence score from highest to lowest. Then, 11 different confidence thresholds (known as ranks) are chosen such that recall at each of those rank values have 11 values ranged from 0 to 1 with 0.1 intervals. These thresholds should be in a way that recalls at those confidence values is $0, 0.1...1.0$. Average precision is now calculated as the average of maximum precision values at those 11 selected recall values. 
\par
Average precision for class \textit{c} is defined as: 
\begin{equation}
    AP_{c} = \frac{1}{11} \sum_{r\in(0,0.1...1)}^R \max(P(r))
\end{equation}
Here $P(r)$ is the precision value for one of the 11 recall values $(r)$

\par

Mean average precision is the average of AP's over all the classes. The mAP is given as:
\begin{equation}
    mAP = \frac{1}{C} \sum_{c}^C AP_{c}
\end{equation}

Here, C is the total number of classes and $AP_{c}$ is the AP for particular class c. 
\afterpage{\null\newpage}
\rhead{\textit{Related work}}
\lhead{\thepage}
\chapter{Related work}

This chapter gives a brief understanding of the evolution of semantic segmentation and object detection solved using \ac{ml} based techniques to \ac{dl} based techniques. This section also gives an insight into the approaches used for semantic segmentation and object detection. Taking inspiration from these approaches, we decide upon the architecture that can be used for this work. 


\subsection{Semantic Segmentation}

Section \ref{Semantic Segmentation} describes the process of linking each pixel of an image with a class. This task is one of the grand challenges in the field of computer vision. It all started with researchers using traditional machine learning algorithm \cite{dollar2009integral} with the help of techniques such as edge detection \cite{huang2010image}, clustering \cite{zheng2018image}, region growing \cite{10.1007/978-3-540-76725-1_21} and SIFT \cite{suga2008object}. But all of these techniques required the features to be extracted manually and hence, this becomes a tedious job and requires domain expertise.
\par
Such \ac{ml}-based techniques slowed down around the era of \ac{dl} as it started to take over the world of computer vision because it needs only data. Also, amongst different learning algorithms in the field of \ac{dl}, \ac{cnn} got a tremendous amount of success in the area of semantic segmentation. 
\par
In \ac{dl}, R-CNN \cite{girshick2014rich} used selective search algorithm \cite{uijlings2013selective} to extract region proposals from the image and then applied \ac{cnn} upon each region proposal and achieved record result for PASCAL VOC dataset. This technique was a leading hand in the area of semantic segmentation at that time. Around the same time \cite{DBLP:journals/corr/GuptaGAM14} used \ac{cnn} along with geocentric embedding on RGB-D images for semantic segmentation. After this phase, \ac{fcn} \cite{long2015fully} gained the highest attention as it achieved \ac{sota} result. \ac{fcn} used base model \ac{vgg}16 as the feature extractor and bilinear interpolation technique for the upsampling of feature maps. It also used skip connections for combining low and high layer features in the final feature map to achieve fine-grained segmentation map. \ac{fcn} used only local information which makes semantic segmentation quite ambiguous. Hence, to reduce the ambiguous information from the image, \cite{DBLP:journals/corr/MostajabiYS14} used contextual features and achieved \ac{sota} result. Lately, \cite{DBLP:journals/corr/RonnebergerFB15} used a U shaped network known as U-Net, which consists of a contracting and expansive pathway approach to semantic segmentation. The contracting path is responsible for extracting image features and reduce spatial information; expansive pathway upsamples the contracted feature map. In each upsampling step, the network concatenates the reduced up-convolved feature map with corresponding cropped feature map from contracting pathway. By using both high level and low-level spatial information, U-Net achieves precise segmentation map. Segnet \cite{DBLP:journals/corr/BadrinarayananK15} also follow the same footsteps of U-Net by using an encoder-decoder network. An encoder is used to extract image features, and the decoder uses un-pooling operation to get a segmentation map of a size similar to the input image in order to get a precise localization of the segmented object. 
\par
This work mainly focuses on encoder-decoder architecture. As mentioned above, the encoder is used to extract features, and give a spatially reduced feature map from images, and the decoder does upsampling of the feature maps to achieve segmented map. 
\clearpage

\subsection{Object Detection}

Section \ref{Object Detection} describes the process of detecting instances of visual objects in an image. This task is one of the most major and challenging problems in the area of computer vision. Nearly all of the early object detection algorithms were based on extracting features manually due to the lack of effective image representations. \cite{viola2001rapid} achieved real-time detection of human faces that is faster than other algorithms and with better detection accuracy. This detector follows the most straight-forward way of sliding windows through all possible scales and locations in an image to check if any of the windows contain a human face. Histogram of Oriented Gradients (HOG) by \cite{dalal2005histograms} was the next revelation in this area, which was considered as an important improvement over other detectors. In order to detect multiple objects of different sizes, the HOG detector re-scales the input image numerous times while keeping the size of the sliding window same. HOG detector has proven to be an important foundation for many object detectors. Such traditional methods became saturated slowly and steadily. 
\par
In 2012, the re-birth of \ac{dl} happened, which lead to the application of \ac{cnn} in the field of object detection. RCNN \cite{DBLP:journals/corr/GuptaGAM14} was the first object detector in the \ac{dl} era. The idea behind RCNN was the extraction of a set of object proposals using selective search \cite{van2011segmentation} and each of these proposals is fed into a \ac{cnn} model to extract features. Lastly, linear SVM classifiers are used to predict the presence of an object within each proposal and identify the class of the object. Fast RCNN \cite{girshick2015fast} detector was an improvement over RCNN that enables to simultaneously train a detector and bounding box regressor under similar network configurations. Faster RCNN by \cite{DBLP:journals/corr/RenHG015} was an upgrade of Fast RCNN. It introduced Region Proposal Network (RPN) that enables complimentary region proposals. Components like feature extraction, bounding box regression, proposal detection were integrated into a unified end-to-end learning framework. After a variety of improvements over the above-mentioned networks, \cite{lin2017feature} proposed a top-down architecture with lateral connections in order to build high-level semantics at all scales. This type of architecture shows great advances for detecting objects at various scales. Above mentioned \ac{dl} methods to detect object are categorized as two-stage detectors where one stage extracts object proposals, and the next stage classifies objects contained in the proposals.  
\par
\ac{dl} methods also consist of one-stage object detectors which skip the object proposal step and runs detection directly over an image. YOLO proposed by \cite{DBLP:journals/corr/RedmonDGF15} is a one-stage detector that divides the input image into grids and predicts bounding boxes and probabilities for each of the grid simultaneously. Due to this, YOLO is speedy at detecting objects. However, YOLO suffers from a drop in localization accuracy for small objects. SSD proposed by \cite{liu2016ssd} has paid attention to this problem. SSD uses multi-reference and multi-resolution detection techniques which makes it better at detecting small objects.
\par
This work mainly focuses on one-stage object detection networks. As mentioned above, one-stage detectors skip the region proposal step and run detection directly. 

\clearpage

\subsection{Combining semantic segmentation and object detection}

The above-mentioned steps give a high-level understanding of the architecture to be used for performing both the tasks. This section involves a brief background as to how both the architectures (semantic segmentation and object detection) can be connected using a single common pipeline. 
\par
\cite{salscheider2019simultaneous} uses an encoder-decoder architecture for both the tasks parallelly. An encoder is a feature extractor whose last feature map is shared in between the semantic segmentation head and object detection head. Also, \cite{DBLP:journals/corr/TeichmannWZCU16} also uses an encoder-decoder architecture where the encoder is shared among 3 tasks, namely, object detection, semantic segmentation and image classification. 

\afterpage{\null\newpage}

\rhead{\textit{Approach and Implementation}}
\lhead{\thepage}

\chapter{Approach and Implementation}

This chapter covers the approach, methodology and design of architecture used in this work. Section \ref{Dataset} gives an explanation about datasets used for training and evaluating the model. Sections \ref{SS} and \ref{OD} explain different architectures using feature extractor as encoders in semantic segmentation and object detection. Section \ref{Combination} shows how we can combine different architectures through a common feature extractor/ encoder that can solve both the problems mentioned above parallelly. 

\subsection{Dataset} \label{Dataset}

Datasets play a vital role in the field of \ac{dl}. A useful dataset gives a significant boost to an algorithm's performance. Preparing a sizeable amount of good and standard datasets require time and knowledge to gather relevant information. However, there are several publicly available standard datasets in the field of image processing.  

\par

Out of all the publicly available datasets, we opt for PASCAL-VOC 2012 \cite{pascal-voc-2012} and KITTI Vision Benchmark \cite{Geiger2012CVPR} datasets to train and evaluate our model. The following sections consist of a brief description of both the datasets mentioned above.

\subsubsection{PASCAL-VOC 2012} \label{PascalVOC}

PASCAL VOC is a benchmark challenge in computer vision tasks, which provides the computer vision and machine learning association with a standard dataset of images, annotations, and evaluation procedures \cite{pascal-voc-2012}. Since it's inception in the year 2005, the dataset consisted of only 1578 images (4 classes: bicycle, car, motorbike, people) for the task of classification and detection; while in the year 2012, the number of images went up more than 10,000 images consisting of 20 classes. The 2012 challenge includes dataset consisting of images that are used for detection, classification and segmentation, which is categorized as:   

\begin{itemize}

    \item Detection and classification - 17, 125 images
    \item Segmentation - 2,913 images 

\end{itemize}

\subsubsection{KITTI Vision Benchmark} \label{Kitti}

\ac{kitti} dataset is a novel challenging real-world computer vision benchmark. This dataset was captured by driving autonomous cars developed by \cite{AnnieWay}. These cars were driven in rural areas and on highways around the mid-size city of Karlsruhe, Germany. Two high-resolution colour and grayscale video cameras attached to these cars are used to capture images for the dataset. Each image in the dataset consists of up to 15 cars and 30 pedestrians. \ac{kitti} consists of images that can be used for detection and segmentation, categorized as:

\begin{itemize}
    \item Detection - 7,481 images
    \item Segmentation - 200 images
\end{itemize}

Some sample images of both the datasets are shown in the table \ref{seg_imgs_pascalvoc} and \ref{seg_imgs_Kitti}. 


\begin{table}[h!]
\caption{Sample images from the PASCAL-VOC 2012 dataset}
\centering
\def\arraystretch{0.25}% 
\setlength\tabcolsep{12pt}
\begin{tabular}{*{3}{m{0.35\linewidth}}}
\hline
\begin{center}\includegraphics[width=4cm, height=4cm]{Dataset_images/2007_000068.jpg}\end{center} & \begin{center}\includegraphics[width=4cm, height=4cm]{Dataset_images/2007_000068.png}\end{center}\\
\hline
\begin{center}\includegraphics[width=4cm, height=4cm]{Dataset_images/2007_000464.jpg}\end{center} & \begin{center}\includegraphics[width=4cm, height=4cm]{Dataset_images/2007_000464.png}\end{center}\\
\hline
\begin{center}\includegraphics[width=4cm, height=4cm]{Dataset_images/2007_001630.jpg}\end{center} & \begin{center}\includegraphics[width=4cm, height=4cm]{Dataset_images/2007_001630.png}\end{center}\\
\hline
\begin{center}\includegraphics[width=4cm, height=4cm]{Dataset_images/2007_002597.jpg}\end{center} & \begin{center}\includegraphics[width=4cm, height=4cm]{Dataset_images/2007_002597.png}\end{center}\\
\hline
\end{tabular}
\label{seg_imgs_pascalvoc}
\end{table}
\clearpage


\begin{table}[h!]
\caption{Sample images from the \ac{kitti} dataset}
\centering
\def\arraystretch{0.25}% 
\setlength\tabcolsep{12pt}
\begin{tabular}{*{3}{m{0.35\linewidth}}}
\hline
\begin{center}\includegraphics[width=4cm, height=4cm]{Kitti/3_10.png}\end{center} & \begin{center}\includegraphics[width=4cm, height=4cm]{Kitti/000003_10.png}\end{center}\\
\hline
\begin{center}\includegraphics[width=4cm, height=4cm]{Kitti/7_10.png}\end{center} & \begin{center}\includegraphics[width=4cm, height=4cm]{Kitti/000007_10.png}\end{center}\\
\hline
\begin{center}\includegraphics[width=4cm, height=4cm]{Kitti/11_10.png}\end{center} & \begin{center}\includegraphics[width=4cm, height=4cm]{Kitti/000011_10.png}\end{center}\\
\hline
\begin{center}\includegraphics[width=4cm, height=4cm]{Kitti/16_10.png}\end{center} & \begin{center}\includegraphics[width=4cm, height=4cm]{Kitti/000016_10.png}\end{center}\\
\hline
\end{tabular}
\label{seg_imgs_Kitti}
\end{table}
\clearpage

This work is based on creating a common pipeline to perform semantic segmentation and object detection. This poses a challenge to create a dataset that contains an equal number of ground truth for semantic segmentation and object bounding box coordinates for object detection. In case of PASCAL-VOC 2012 dataset, the bounding box coordinates for each object in the image are already available in XML format, but it's not the same in the case of \ac{kitti} dataset. Hence, a possible solution to this is using \cite{LabelImg} to create bounding box coordinates for each object manually and save it in XML format. 

\par

Figure \ref{LabelImg} shows how bounding box coordinates can be created using LabelImg tool. 

\begin{figure}[h!]
    \centering
    \includegraphics[width=12cm]{Kitti/LabelImg.png}
    \caption{LabelImg tool}
    \label{LabelImg}
\end{figure}


Training-validation split for both the datasets are: 

\begin{itemize}
    \item PASCAL-VOC 2012 - Training: 2310 Validation: 583 
    \item \ac{kitti}  - Training: 160 Validation: 40
\end{itemize}

\subsection{Data augmentation} \label{augmentations}

Data augmentation is a strategy that enables the diversity in data available for training \ac{dl} models. This strategy proves to be important because it helps reducing overfitting and improves the generalization of the model. In order to achieve this, a series of augmentations on the data can be applied. Augmentation of data can either be performed offline or on the fly such that it can cover most of the variances and scenarios expected to appear in the real world. 
\par
Some of the augmentations applied randomly to the data in our work are:
\begin{itemize}
    \item Brightness transform: $-30$ to $30$
    \item Contrast transform : $0.5$ to $1.5$
    \item Hue transform: $-18$ to $18$
    \item Saturation transform: $0.5$ to $1.5$
    \item Horizontal flip
\end{itemize}


\subsection{Feature extractor/ Encoder} \label{Encoder}

Section \ref{feature_extractor} mentions how \ac{cnn} is used to extract features from images. As this approach is based on using a common encoder to do semantic segmentation and object detection, choosing an effective image feature extractor is of utmost importance. \cite{simonyan2014very} neural network from VGG group, University of Oxford is used for extracting image features in this work. This architecture is an improvement over \cite{krizhevsky2012imagenet} where large-sized filters are replaced with multiple small-sized filters stacked one after the other and increasing the depth of the network. The arrangement of the layers are consistent and built using $3\times3$ convolutional layers with stride 1 to increase the depth, and the volume is reduced by using max-pooling of $2\times2$ filter with stride 2. In the end, 2 fully connected layers, each with 4,096 nodes, followed by a softmax function for the output. 


% **From the input layer to the last max pooling layer (labeled by 7 x 7 x 512) is regarded as the feature extraction part of the model, while the rest of the network is regarded as the classification part of the model.**

\par
Some pros and cons using \ac{vgg}16 are stated below:
\par
Pros:
\begin{itemize}
    \item Multiple, stacked,  smaller sized filters increases the depth of network hence, enabling it to learn more complex features at a lower cost. 
    \item Very simple and consistent architecture.
\end{itemize}
\par
Cons:
\begin{itemize}
    \item Consists of too many weight parameters making it very memory consuming.
    \item Too many parameters makes the training process slow.
\end{itemize}

\par
Figure \ref{VGG} shows the architecture of \ac{vgg}16. 

\begin{figure}[h!]
    \centering
    \includegraphics[width =\linewidth]{Architectures/VGG.png}
    \caption{VGG16}
    \label{VGG}
\end{figure}

\subsection{Semantic segmentation} \label{SS}

Section \ref{Semantic Segmentation} gives an understanding as to how semantic segmentation is performed using an encoder-decoder architecture. \ac{fcn} \cite{long2015fully} is used for semantic segmentation in this work. \ac{fcn} has revolutionized the area of semantic segmentation as it can process the whole image in a single forward pass and hence, exploits contextual information in images efficiently and better way. \ac{fcn} is an encoder-decoder architecture, where, \ac{vgg}16 is used as a pre-trained encoder (feature extractor). This architecture consists of only convolutional and pooling layers, giving them the ability to produce arbitrary sized predictions. Due to this reason, the fully connected layers (Fc6 and Fc7) of \ac{vgg}16 are converted to $1\times1$ convolution. The extracted features are upsampled using transposed convolutions using bilinear interpolation features. Skip connection is introduced after each convolution block to merge features from various resolution levels that helps in combining context information with spatial information. Figure \ref{FCN_Architecture} shows the architecture diagram of \ac{fcn}. 
 
\begin{figure}[h!]
    \centering
    \includegraphics[width=\linewidth]{Architectures/FCN_Archi.jpg}
    \caption{Architecture of \ac{fcn}-32, \ac{fcn}-16 and \ac{fcn}-8}
    \label{FCN_Architecture}
\end{figure}

There are 3 variants in \ac{fcn}. These variants have common downsampling path but different respective upsampling paths; they are:

\begin{itemize}

\item \ac{fcn}-32: Segmentation map is directly created from conv7 layer by using a transposed convolution layer with stride 32.
    
\item \ac{fcn}-16: The prediction from conv7 is upsampled using the transposed convolution of stride 2 (2x upsampling) and added to the pool4 feature map. This summed feature map is upsampled again using the transposed convolution of stride 16 to produce segmentation map. 


% Sums the 2x upsampled prediction from conv7 (using a transposed convolution with stride 2) with pool4 and then produces the segmentation map, by using a transposed convolution layer with stride 16 on top of that.
    
    
\item \ac{fcn}-8: The summed feature map of conv7 (2x) and pool4 is upsampled using the transposed convolution of stride 2 and added to pool3 feature map. This summed feature map is upsampled using transposed convolution with stride 8 to produce segmentation map. 


% Sums the 2x upsampled conv7 (with a stride 2 transposed convolution) with pool4, upsamples them with a stride 2 transposed convolution and sums them with pool3, and applies a transposed convolution layer with stride 8 on the resulting feature maps to obtain the segmentation map.

\end{itemize}


\subsubsection{Calculating loss}

\paragraph{Pixel-wise cross entropy loss}

This is one of the most generally used loss function for the task of semantic segmentation. It compares each pixel of the class prediction (depth-wise pixel vector) to the one-hot encoded class label. 
Figure \ref{crossentropy} gives an example of how loss is calculated.
\par
Pixel-wise cross-entropy loss can be calculated as: 

% The most commonly used loss function for the task of image segmentation is a pixel-wise cross entropy loss. This loss examines each pixel individually, comparing the class predictions (depth-wise pixel vector) to our one-hot encoded target vector.

\begin{equation}
    -\Sigma_{classes}\;y_{true}\;log(y_{pred})
\end{equation}


\begin{figure}[h!]
    \centering
    \includegraphics[width=\linewidth]{Segmentation_images/crossentropy.png}
    \caption{(Left) Prediction for a selected pixel and (Right) one hot vector of the corresponding pixel}
    \label{crossentropy}
\end{figure}

\par Pixel-wise loss is calculated as the log loss and is summed over all classes contained in the dataset. This calculation is averaged after being calculated for all pixels.

\newpage

\subsection{Object Detection} \label{OD}

Section \ref{Object Detection} gives an understanding of the approaches of object detection and how one stage object detector works. \ac{ssd} \cite{liu2016ssd} is used for object detection in this work. \ac{ssd} is a one stage detector that learns to map classification and regression problem directly from the raw image to bounding box coordinates in a single global step, hence the name 'single shot'. Figure \ref{SSD_Architecture} shows the \ac{ssd} architecture

\begin{figure}[h!]
    \centering
    \includegraphics[width=\linewidth]{Architectures/ssd.jpeg}
    \caption{Architecture of \ac{ssd}}
    \label{SSD_Architecture}
\end{figure}

\ac{ssd} network is composed of six major parts that play an important role in detection objects; they are: 

\begin{itemize}
    \item Feature extractor
    \item Convolutional box predictor
    \item Anchors generation
    \item Matching priors to actual bounding box coordinates
    \item Hard negative mining
    \item Non-maximum suppression as a post-processing step.
\end{itemize}

\paragraph{Feature extractor}

\ac{ssd} is based on \ac{vgg}16 which works as a feature extractor. 

\paragraph{Convolutional box predictor}

For object detection, last fully connected layers of \ac{vgg}16 are discarded, and a set of auxiliary convolution layers are added which are called as convolutional box predictor. These layers are responsible in producing multiple feature maps of shape $19\times19$, $10\times10$, $5\times5$, $3\times3$ and $1\times1$ in successive order stacked after the feature map coming from last convolutional layer from \ac{vgg}16 of size $38\times38$. These six auxiliary layers extract features from the image at multiple scales and progressively decrease the spatial resolution. Due to performing convolution operation at multiple scales, objects of various sizes can be detected. For example, feature maps of $10\times10$ perform better in detecting smaller objects while feature maps of size $3\times3$ perform better in detecting large objects. Each of these layers is connected to two heads, one that performs regression to find the bounding boxes and the other one to a classifier to classify each of the detected boxes.

\begin{figure}[h!]
    \centering
    \includegraphics[width=\linewidth]{Architectures/boxpredictor.png}
    \caption{Auxiliary box predictor layers \cite{AI-Dairy}}
    \label{Auxiliary}
\end{figure}

\paragraph{Anchors generation}

Anchors generation is one of the most critical steps that can significantly affect the overall performance of the model. Anchors are a set of pre-defined boxes overlayed on the input as well as the generated feature maps at different spatial locations, scales and aspect ratios that act as reference points for the model to predict ground truth boxes. Anchors are significant because they provide a strong starting point for the regressor to predict ground truth boxes rather than starting to predict with random co-ordinates. Each feature map is divided into several grids, and each grid is associated with a set of default anchor bounding boxes of different dimensions and aspect ratios. Each of the priors is responsible for predicting exactly one bounding box.  

\begin{figure}[h!]
    \centering
    \includegraphics[width=\linewidth]{Obj_det_images/anchors.png}
    \caption{Anchors overlayed on $5\times5$ map \cite{AI-Dairy}}
    \label{Anchors}
\end{figure}

Figure \ref{Anchors} shows each grid of a $5\times5$ feature map having 6 anchors which mostly matches the objects that need to be detected. For example, the red coloured anchors are most likely to detect the car, whereas blur coloured anchors are likely to detect the person and car of both smaller and larger sizes. 


\paragraph{Matching priors to actual bounding box co-ordinates}

Each ground truth bounding box coordinates is matched to anchors having the largest Intersection over Union (IoU) or Jaccard index. Also, unassigned anchors having IoU greater than a threshold (usually 0.5) are also matched to the ground truth boxes. All matched anchors are called positive samples, and the rest are considered as negative samples. 

\paragraph{Hard negative mining}

Among a large number of anchors, only a few will be considered as positive samples, and the rest of the anchors will be considered as negative samples. Hence, from a training point of view, this imbalance between the positive and negative samples will make the training biased towards negatives. To avoid such a situation, some of the negative samples are randomly sampled from the negative pool such that the training set has a negative to positive ratio as 3:1. This process is known as hard negative mining. This reason to keep the negative samples is to let the network know about the features that lead to incorrect detection. 

\paragraph{Non-maximum suppression}

As explained in section \ref{Object Detection}, non-maximum suppression is used to attain top predictions for each class. This makes certain that only the most probable predictions are kept by the model, while the noisy ones are discarded. 
\par

As \ac{ssd} is a single-shot object detector, all of these components are trained simultaneously. 

\subsubsection{Calculating loss}

The overall loss function is a weighted sum of localization loss (loc) and the confidence loss (conf) where localization loss is the mismatch between ground truth box and predicted bounding box. Confidence loss is the loss in assigning class labels to the predicted bounding boxes. The equation of the loss function is given as: 

\begin{equation}
    L(x,c,l,g) = \frac{1}{N} (L_{conf} (x,c) + \alpha L_{loc} (x,l,g))
\end{equation}

Here x = 1 if the anchor matches the determined ground truth box, and 0 otherwise. N is the number of matched anchors, l is predicted bounding box, g is ground truth box, c is class, $L_{conf}$ and $L_{loc}$ are confidence and localization loss. $\alpha$ is the weight for localization loss. Smooth-L1 loss \cite{Girshick_2015_ICCV} is used for localization on \textit{l} and \textit{g}, softmax loss is used for optimization of confidence loss over multiple class confidence \textsc{c}

\subsubsection{L2 regularization}

This is a method used to reduce model complexity and overfitting on training data. During the phase of training, a handful of weights can get too large and thus influence the prediction heavily by suppressing the rest of the weights. This makes the model lose the power to generalize and in turn, overfit on training data. Hence, this method penalizes such large weights and makes sure the model learns smaller and consistent weights, through-out all the layers. 

\begin{equation}
    L = \frac{1}{N} \sum_{i=1}^{N} L_{i} + \lambda  \sum_{i} \sum_{j} W_{i,j}^2
\end{equation}

Here, N is the number of classes in the dataset, $L_{i}$ is a loss for particular class \textit{i}. 
$W_{i,j}^2$ is the regularization function that calculates the sum of squares of each weight from the weight matrix W and $\lambda$ is the regularization parameter which specifies the strength of regularization.


\subsubsection{Batch Normalization}

Batch normalization \cite{ioffe2015batch} is also a regularization technique that accelerates the model training by allowing it to use higher learning-rates. During the training phase, a large dataset is passed as batches to the model. The distribution of the batches usually differ from each other, and this affects the behaviour of the model. This change is distribution is known as covariate shift. Due to covariate shift, layer activations change quickly to adapt to changing inputs. As activations of the previous layer are passed as output to the next layer, all layers of the model get affected inconsistently for each batch. This results in a long time for the model to converge. To avert this, activations of each layer are normalized to have zero mean and unit variance. This helps each layer to learn the more stable distribution of inputs in all batches, which speeds up the model training. However, forcing activations to be fixed can limit the representation of input data. Instead, batch normalization allows the model to learn parameters $\beta$ and $\gamma$ during the training phase to convert the mean and variance according to the input data. 

\begin{equation}
    y^(k) = \gamma^{(k)} \hat x^{(k)} + \beta^{(k)}
\end{equation}

$\hat x^{(k)}$ is a single activation and $\gamma^{(k)}$, $\beta^{(k)}$ are parameters that scale and shift the normalized value during training. 

\clearpage


\subsection{Combining architectures} \label{Combination}
\vfill
\begin{figure}[H]
   \centering
    \includegraphics[width= \linewidth]{Architectures/encoder1.png}
    \caption{Combined architecture}
    \label{combination}
\end{figure} 
\vfill
\clearpage

\subsubsection{Combined architecture} \label{combinearchi}

Sections \ref{SS} and \ref{OD} give an understanding on how semantic segmentation and object detection work separately. Figure \ref{Combination} shows how both the architectures can be combined using a common feature extractor/ encoder. The combined architecture uses \ac{vgg}16 as the common encoder and transfers the image features to both the branches of the architecture. 'Branch1' is the \ac{ssd} network that performs object detection and 'Branch2' is the \ac{fcn} network that performs semantic segmentation. 
\par
'Branch1' consists of six auxiliary convolutional layers responsible for producing feature maps of shape $19\times19$, $10\times10$, $5\times5$, $3\times3$ and $1\times1$ in succession stacked after the last feature map produced by \ac{vgg}16 encoder. These layers further extract image features from the image at multiple scales such that objects of different sizes can be detected. For example, feature maps of $3\times3$ are useful to detect big sized objects. Each of these layers is connected to a regressor and a classifier. Regressor predicts the bounding boxes for objects present in the image. Anchors play an important role for the regressors to work. Anchors are a set of pre-defined boxes that are overlayed on the input and the feature maps at different locations, scales and aspect ratios. They provide a starting point for the regressor to predict ground-truth boxes.  Once the bounding boxes are predicted, the classifier predicts the class of the object contained in the predicted bounding box. Lastly, non-maximum suppression is applied to keep top predictions and remove the noisy ones. 

\par
'Branch2' consists of transposed convolutional layers responsible for producing segmentation maps from the last feature map of \ac{vgg}16 encoder. These transposed convolutional layers upsample the feature maps to the original size as the input. Upsampling can be done in 3 ways: 
There are 3 ways to upsample the feature map. One way is to upsample the feature map directly by applying the transposed convolution to produce the segmentation map. The other two techniques involve skipping connections to merge features from different resolution levels and applying transposed convolution to produce a segmentation map.   


\paragraph{Combined loss} \label{combineloss}

Once both the architectures are combined, the respective losses also need to be combined such that the network can learn both detection and segmentation parallelly. 'Branch1' consists of two losses, localization loss and confidence loss. Localization loss measures the mismatch between ground truth bounding box coordinates and predicted bounding box coordinates. Confidence loss measures the mismatch between ground truth class labels and predicted class labels inside the predicted bounding box. Similarly, 'Branch2' consists of pixel-wise cross-entropy loss. This loss calculates the difference between the predicted vector and one hot encoded class label. The combined loss for our architecture is given as: 

\begin{equation}
    L = \frac{1}{N} (L_{conf} (x,c) + \alpha L_{loc} (x,l,g)) + (-\Sigma_{classes}\;y_{true}\;log(y_{pred}))
\end{equation}

The left part of the loss calculates the localization and confidence loss used for object detection while the right side calculates the pixel-wise loss, which is used for semantic segmentation.  
\par

N is the number of correctly matched anchors with the ground truth bounding box. Value of x=1 if the anchor matches with the ground truth bounding box and 0 if it doesn't match. l is the predicted bounding box while g is the ground truth bounding box. $y_{true}$ is the actual ground truth of semantic segmentation and $y_{pred}$ is the prediction from the model.

\clearpage



\afterpage{\null\newpage}



\rhead{\textit{Experiments and Results}}
\lhead{\thepage}

\chapter{Experiments and Results}

In this chapter, experiments and results of standalone architectures and combined architectures are discussed. Section \ref{pascalvoc} is about experiments using PASCAL-VOC 2012 dataset followed by section \ref{kitti} about experiments using \ac{kitti} dataset.

\subsection{Evaluation on PASCAL-VOC 2012} \label{pascalvoc}

Section \ref{combinearchi} shows how combining the architectures (\textit{Branch1} and \textit{Branch2}) is possible using a common encoder. Before evaluating the combined architecture, it is necessary to check the performance evaluation of each branch. This technique is useful to know the flaw or benefit of using the combined architecture in comparison to using stand-alone architectures.

\par

\paragraph{Hyperparameters}

Hyperparameters are kept constant throughout all experiments to get an ideal comparison. 

\begin{table}[h!]
\centering
\begin{tabular}{|c|c|}
\hline
\textbf{Hyperparameter} & \textbf{Value}                 \\ \hline
Learning rate  & 0.0001; 0.001; 0.0001 \\ \hline
Intervals      & 18,300; 36,600        \\ \hline
Batch size     & 8                     \\ \hline
Weight decay   & 0.000001              \\ \hline
Epochs         & 200                   \\ \hline
\end{tabular}
\end{table}

\paragraph{Branch1 - Standalone - Object detection}

\textit{Branch1} - standalone only involves training of \textit{Branch1} while the \textit{Branch2} is freezed. During this phase, only the localization and confidence loss is calculated. Images are resized to $300 \times 300$, and random augmentation is applied on the fly. 

\par

This setup gives an mAP score of \textbf{46.9\%}
\par

\begin{figure}[h!]
    \centering
    \includegraphics[width = \linewidth]{results/Onlydet.png}
    \caption{Loss vs epochs - Branch1}
    \label{onlyobjcurve}
\end{figure}


Figure \ref{onlyobjcurve} shows the loss vs epochs curve for standalone object detection. It is seen that after the $120^{th}$ epoch, the loss increases neither decrease, which means the network has saturated.  

Some sample predictions are shown in table \ref{Onlyobjdet}

\clearpage

\begin{table}[h!]
\caption{Predicted images from Branch1 - standalone - object detection}
\centering
\def\arraystretch{0.25}% 
\setlength\tabcolsep{12pt}
\begin{tabular}{*{3}{m{0.35\linewidth}}}
\hline
\begin{center}\includegraphics[width=4cm, height=4cm]{Onlydetection/Bus.png}\end{center} & \begin{center}\includegraphics[width=4cm, height=4cm]{Onlydetection/cat.png}\end{center}\\
\hline
\begin{center}\includegraphics[width=4cm, height=4cm]{Onlydetection/cow.png}\end{center} & \begin{center}\includegraphics[width=4cm, height=4cm]{Onlydetection/horse.png}\end{center}\\
\hline
\begin{center}\includegraphics[width=4cm, height=4cm]{Onlydetection/chair.png}\end{center} & \begin{center}\includegraphics[width=4cm, height=4cm]{Onlydetection/pottedplant.png}\end{center}\\
\hline
\end{tabular}
\label{Onlyobjdet}
\end{table}


\paragraph{Branch2 - Standalone - semantic segmentation}

\textit{Branch2} - standalone only involves the training of \textit{Branch2} keeping the \textit{Branch1} freezed. During this phase only the pixel-wise cross entropy is calculated. The predicted segmentation map is achieved by upsampling the last feature map of the \ac{vgg}16 encoder by 32 times (\ac{fcn}32). Images are resized to $300 \times 300$ and random augmentation is applied on the fly.

\par

This setup gives a mean IoU score of \textbf{62.01\%} and mean pixel accuracy score of \textbf{85.4\%}.

\clearpage

\begin{figure}[h!]
    \centering
    \includegraphics[width = \linewidth]{results/Fcn32-Onlyseg.png}
    \caption{Loss vs epochs - Branch2}
    \label{onlysegcurve}
\end{figure}

Figure \ref{onlysegcurve} shows the loss vs epochs curve for semantic segmentation. The loss value at the end of $200^{th}$ epoch is \textbf{1.04}. Also, it is seen that after $65^{th}$ epoch we reach the minima of function where the value of loss is \textbf{0.54}.

Some sample predictions are shown in table 
\ref{Onlyseg}

\begin{table}[h!]
\caption{Predicted images from Branch2- standalone - semantic segmentation
(Extreme left - RGB image, Bottom - ground truth, Extreme right - prediction)}
\centering
\def\arraystretch{0.25}% 
\setlength\tabcolsep{10pt}
\begin{tabular}{*{3}{m{0.20\linewidth}}}
\hline
\begin{center}\includegraphics[width=3cm, height=3cm]{Onlysegmentation/2011_002040.jpg}\end{center} &
\begin{center}\includegraphics[width=3cm, height=3cm]{Onlysegmentation/20_gt.png}\end{center} & \begin{center}\includegraphics[width=3cm, height=3cm]{Onlysegmentation/20_pred.png}\end{center}\\
\hline
\begin{center}\includegraphics[width=3cm, height=3cm]{Onlysegmentation/2007_007432.jpg}\end{center} &
\begin{center}\includegraphics[width=3cm, height=3cm]{Onlysegmentation/140_gt.png}\end{center} & \begin{center}\includegraphics[width=3cm, height=3cm]{Onlysegmentation/140_pred.png}\end{center}\\
\hline
\begin{center}\includegraphics[width=3cm, height=3cm]{Onlysegmentation/2009_002164.jpg}\end{center} &
\begin{center}\includegraphics[width=3cm, height=3cm]{Onlysegmentation/300_gt.png}\end{center} & \begin{center}\includegraphics[width=3cm, height=3cm]{Onlysegmentation/300_pred.png}\end{center}\\
\hline
\end{tabular}
\label{Onlyseg}
\end{table}

\clearpage

\paragraph{Combining Branch1 and Branch2 (FCN32)}

This phase involves the training of both the branches such that one branch gives the object detection result and the other branch give segmentation map. The predicted segmentation map is achieved by upsampling the last feature map of the \ac{vgg}16 encoder by 32 times (\ac{fcn}32). Localization, confidence and pixel-wise cross-entropy are calculated in this phase. Images are resized to $300 \times 300$, and random augmentation is applied on the fly.

\par

Training both the branches parallelly gives mean AP score of \textbf{46.6\%} for object detection and mean IoU score of \textbf{56.85\%}, mean pixel accuracy score of \textbf{83.3\%} for semantic segmentation.

\par

\begin{figure}[h!]
    \centering
    \includegraphics[width = \linewidth]{results/combined.png}
    \caption{Loss vs epochs}
    \label{combined}
\end{figure}

Figure \ref{combined} shows the loss vs epochs curve for the combined training. As per the curve, the loss is increasing gradually at a high rate after $80^{th}$ epoch. 

\begin{figure}[h!]
    \centering
    \includegraphics[width = \linewidth]{results/combined-training.png}
    \caption{Loss vs epochs (Training)}
    \label{combined_training_fig}
\end{figure}

Figure \ref{combined_training_fig} shows the training loss vs epochs curve. By comparing both the above graphs, we can deduce that the network has overfitted on training data. 

Hence, scores at the $80^{th}$ are reported as network starts to overfit after that. The mean AP score for object detection is \textbf{35.65\%} while the mean IoU score and mean pixel-accuracy score are \textbf{46.92\%} and \textbf{80.1\%} for semantic segmentation. 

% By combining the architecture, the mean AP score drops down by \textbf{11.25\%} for object detection and mean IoU score and mean pixel-accuracy score drops down by \textbf{15.09\%} and \textbf{4.84\%} in case of semantic segmentation. 

Some sample predictions are shown in table \ref{combinedpred}
\ref{combinedpred}

\newpage

\begin{table}[h!]
\caption{Predicted images from combined architecture - object detection + semantic segmentation}
% (Extreme left - RGB image, Bottom - ground truth, Extreme right - prediction)}
\centering
\def\arraystretch{0.25}% 
\setlength\tabcolsep{10pt}
\begin{tabular}{*{3}{m{0.20\linewidth}}}
\hline
\begin{center}\includegraphics[width=3cm, height=3cm]{Combined/person.png}\end{center} &
\begin{center}\includegraphics[width=3cm, height=3cm]{Combined/pottedplant.png}\end{center} & \begin{center}\includegraphics[width=3cm, height=3cm]{Combined/sofa.png}\end{center}\\
\hline
\begin{center}\includegraphics[width=3cm, height=3cm]{Combined/2009_003071.jpg}\end{center} &
\begin{center}\includegraphics[width=3cm, height=3cm]{Combined/140_gt.png}\end{center} & \begin{center}\includegraphics[width=3cm, height=3cm]{Combined/140_pred.png}\end{center}\\
\begin{center}\includegraphics[width=3cm, height=3cm]{Combined/2007_003022.jpg}\end{center} &
\begin{center}\includegraphics[width=3cm, height=3cm]{Combined/460_gt.png}\end{center} & \begin{center}\includegraphics[width=3cm, height=3cm]{Combined/460_pred.png}\end{center}\\
\begin{center}\includegraphics[width=3cm, height=3cm]{Combined/2010_005429.jpg}\end{center} &
\begin{center}\includegraphics[width=3cm, height=3cm]{Combined/360_gt.png}\end{center} & \begin{center}\includegraphics[width=3cm, height=3cm]{Combined/360_pred.png}\end{center}\\
\hline
\end{tabular}
\label{combinedpred}
\end{table}

\paragraph{Combining Branch1 and Branch2 using Lambda in loss function - (FCN32)}

Section \ref{combineloss} shows how the combined loss is calculated for semantic segmentation and object detection. This experiment shows how using $\lambda$ in the combined loss function will work with the model. 
\begin{equation}
    Combined loss = \lambda (loss of Branch1) + \lambda (loss of Branch2)
\end{equation}
\par
The combined architecture can be seen as a multi-task setting where object detection and semantic segmentation are being solved with the help of common parameters and very few task-specific parameters. This idea of using a combined loss function then can be seen as a generalization strategy, where one of the above-mentioned tasks put pressure on the common parameters such that the other task can be generalized better. Of course, in an ideal scenario where the two tasks are highly related, adding the loss functions with $\lambda = 1$ would suffice. However, by adding a small value of $\lambda$ for one of the tasks will add as a regularizer (generalizes the parameters by putting some constraints) for the other task.
\par
The predicted segmentation map is achieved by upsampling the last feature map of the \ac{vgg}16 encoder by 32 times (\ac{fcn}32). Images are resized to $300 \times 300$, and random augmentation is applied on the fly.

\begin{itemize}
\item Case1: $\lambda_{1}=1$ and $\lambda_{2}$ = 0.01
    
\begin{equation}
    L = \lambda_{1} (-\Sigma_{classes}\;y_{true}\;log(y_{pred})) + \lambda_{2}(\frac{1}{N} (L_{conf} (x,c) + \alpha L_{loc} (x,l,g))) 
\end{equation}
As per the loss function above, the object detection part will act as a regularizer for the semantic segmentation part. 

\par

This setup of $\lambda$ value gives a mean IoU score of \textbf{62.52\%} and mean pixel-accuracy score of \textbf{86.3\%} for semantic segmentation. The mAP score for object detection drops down to only \textbf{3\%}.

\par

\begin{figure}[h!]
    \centering
    \includegraphics[width=\linewidth]{results/lambdaseg.png}
    \caption{Semantic segmentation loss vs epochs}
    \label{labdaseg}
\end{figure}

Figure \ref{labdaseg} shows the loss vs epochs curve. It is seen that loss is increasing after the $70^{th}$ epoch.


\begin{figure}
\centering
    \includegraphics[width=\linewidth]{results/traininglambdaseg.png}
    \caption{Semantic segmentation loss vs epochs (Training)}
    \label{labdasegtraining}
\end{figure}

Figure \ref{labdasegtraining} shows the training loss vs epochs curve. By comparing both the above graphs, we can deduce that the network has overfitted on training data. 

\newpage

Hence, scores at $70^{th}$ epoch are reported as network starts to overfit after that. The mean
AP score for object detection is \textbf{0.37\%} while the mean IoU score and mean pixel-accuracy for semantic segmentation are \textbf{48.65\%} and \textbf{79.82\%}. 


\item Case2: $\lambda_{1}=0.01$ and $\lambda_{2}$ = 1

\begin{equation}
    L = \lambda_{1} (-\Sigma_{classes}\;y_{true}\;log(y_{pred})) + \lambda_{2}(\frac{1}{N} (L_{conf} (x,c) + \alpha L_{loc} (x,l,g))) 
\end{equation}

As per the loss function above, the semantic segmentation part will act as a regularizer for the object detection part. This setup of $\lambda$ value gives a mean AP score of \textbf{45.02\%} while mean IoU and mean pixel-accuracy scores for semantic segmentation are \textbf{32.48\%} and \textbf{73.7\%}.

\begin{figure}[h!]
\begin{multicols}{2}
    \includegraphics[width=\linewidth]{results/lambdaod-confidence.png}\par \includegraphics[width=\linewidth]{results/lambdaod-localization.png}\par
    \end{multicols}
\caption{Confidence and localization loss vs epochs}
\label{conf_loc}
\end{figure}

\begin{figure}[h!]
\centering
    \includegraphics[width=\linewidth]{results/lambdaod-total.png}
    \caption{Total loss vs epochs}
    \label{totallosslambda}
\end{figure}

\end{itemize}

Figures \ref{conf_loc} and \ref{totallosslambda} are confidence, localization and total loss curves. It shows that the total loss gradually increases at a high rate after the $76^{th}$ epoch. The reasons for increase in total loss can be because of the confidence loss increasing and also might be because the network is overfitting the training data. 

Hence, scores from the $77^{th}$ epoch are reported. The mean AP score for object detection is \textbf{36.46\%} while the mean IoU and mean pixel-accuracy scores are \textbf{32.48\%} and \textbf{73.7\%}.

\paragraph{Combining Branch1 and Branch2 - (FCN32)}

This experiment is carried out by resizing the images to $512\times512$ and passing it to the network. The predicted segmentation map is achieved by upsampling the last feature map of the \ac{vgg}16 encoder by 32 times (\ac{fcn}32). Augmentation of images are done on the fly.

Training both the branches gives mean AP score of \textbf{47.66\%} for object detection while mean IoU and mean pixel-accuracy scores of \textbf{61.5\%} and \textbf{47.6\%} for semantic segmentation. 

\begin{figure}[h!]
\centering
    \includegraphics[width=\linewidth]{results/512.png}
    \caption{Total loss vs epochs}
    \label{512feature}
\end{figure}

Figure \ref{512feature} shows the loss vs epochs curve. It is seen that loss is increasing after the $77^{th}$ epoch.

\begin{figure}[h!]
\centering
    \includegraphics[width=\linewidth]{results/512train.png}
    \caption{Total loss vs epochs (Training)}
    \label{512featuretrain}
\end{figure}

Figure \ref{512featuretrain} shows the training loss vs epoch. By comparing both the above graphs, we can deduce that the network has overfitted on training data. 
\par
Hence, scores at $77^{th}$ epoch are reported. The mean AP is \textbf{38.39\%} for object detection while mean IoU score and mean pixel accuracy scores are \textbf{49.96\%} and \textbf{80.3\%}.



\paragraph{Combining Branch1 and Branch2 - (FCN8)}

This experiment is carried out by resizing the images to $300\times300$ and passing it to the network. Augmentation of images are done on the fly. The predicted segmented map is achieved by upsampling and using skip connections as mentioned in section \ref{SS}

Training both the branches parallelly gives mean AP score of \textbf{0.6\%}for object detection while mean IoU and mean pixel-accuracy of \textbf{32.48\%} and  \textbf{73.7\%}for semantic segmentation.

\clearpage


\newpage

\paragraph{Summary of results for PASCAL-VOC}

\begin{table}[h!]
\begin{tabular}{|c|c|c|c|}
\hline
\multicolumn{4}{|c|}{\textbf{Standalone}} \\ \hline
 &
  \textbf{mAP} &
  \textbf{mIoU} &
  \textbf{mPA} \\ \hline
Branch1 (object detection) (300 x 300) &
  46.9\% &
  - &
  - \\ \hline
Branch2 (semantic segmentation) (300 x 300) &
  - &
  62.01\% &
  85.4\% \\ \hline
\multicolumn{4}{|c|}{\textbf{Combined}} \\ \hline
 &
  \textbf{mAP} &
  \textbf{mIoU} &
  \textbf{mPA} \\ \hline
Combined (FCN32) (300 x 300) &
  35.65\% &
  46.92\% &
  80.1\% \\ \hline
\begin{tabular}[c]{@{}c@{}}Combined [(($\lambda$=1)segmentation loss)+\\ (($\lambda$=0.01)detection loss)] - (FCN32)) (300 x 300)\end{tabular} &
  0.37\% &
  48.65\% &
  79.82\% \\ \hline
\begin{tabular}[c]{@{}c@{}}Combined [(($\lambda$=0.01)segmentation loss)+\\ (($\lambda$=1)detection loss (FCN32))] (300 x 300)\end{tabular} &
  36.46\% &
  32.48\% &
  73.7\% \\ \hline
Combined (FCN8) (300 x 300) &
   0.6\%&
   32.48\%&
   73.7\%\\ \hline
Combined (FCN32) (512 x 512) &
  \textbf{38.39\%} &
  \textbf{49.96\%} &
  \textbf{80.3\%} \\ \hline
\end{tabular}
\end{table}

\newpage
\subsection{Evaluation on \ac{kitti}} \label{kitti}

\paragraph{Hyperparameters}

\begin{table}[h!]
\centering
\begin{tabular}{|c|c|}
\hline
\textbf{Hyperparameter} & \textbf{Value}                 \\ \hline
Learning rate  & 0.0001\\ \hline
Batch size     & 4                     \\ \hline
Weight decay   & 0.000001              \\ \hline
Epochs         & 50                   \\ \hline
\end{tabular}
\end{table}

\paragraph{Branch1 - Standalone - Object detection}

\textit{Branch1} - standalone only involves training of \textit{Branch1} while the \textit{Branch2} is freezed. Only the localization and confidence loss is calculated in this phase. Images are resized to $300\times300$ and random augmentation is applied on the fly.

\par

This setup gives an mAP score of \textbf{35.95\%}

\begin{figure}[h!]
    \centering
    \includegraphics[width=\linewidth]{results/k-od.png}
    \caption{Loss vs epoch}
    \label{k-od}
\end{figure}

Figure \ref{k-od} shows the loss vs epochs curve. 

\par

Some sample predictions are shown in table \ref{k-Onlyobjdet}

\clearpage

\begin{table}[h!]
\caption{Predicted images from Branch1 - standalone - object detection}
\centering
\def\arraystretch{0.25}% 
\setlength\tabcolsep{12pt}
\begin{tabular}{*{3}{m{0.35\linewidth}}}
\hline
\begin{center}\includegraphics[width=4cm, height=4cm]{k-Onlydetection/1.png}\end{center} & \begin{center}\includegraphics[width=4cm, height=4cm]{k-Onlydetection/2.png}\end{center}\\
\hline
\begin{center}\includegraphics[width=4cm, height=4cm]{k-Onlydetection/3.png}\end{center} & \begin{center}\includegraphics[width=4cm, height=4cm]{k-Onlydetection/4.png}\end{center}\\
\hline
\begin{center}\includegraphics[width=4cm, height=4cm]{k-Onlydetection/5.png}\end{center} & \begin{center}\includegraphics[width=4cm, height=4cm]{k-Onlydetection/6.png}\end{center}\\
\hline
\end{tabular}
\label{k-Onlyobjdet}
\end{table}

\paragraph{Branch2 - Standalone - Semantic segmentation}

\textit{Branch2} - standalone only involves the training of \textit{Branch2} keeping the \textit{Branch1} freezed. During this phase only the pixel-wise cross entropy is calculated. The predicted segmentation map is achieved by upsampling the last feature map of the \ac{vgg}16 encoder by 32 times (\ac{fcn}32). Images are resized to $300\times300$ and random augmentation is applied on the fly.

\par

This setup gives a mean IoU score of \textbf{60.83\%} and mean pixel accuracy score of \textbf{95.6\%}. 

\begin{figure}[h!]
    \centering
    \includegraphics[width=\linewidth]{results/k-seg.png}
    \caption{Loss vs epoch}
    \label{k-seg}
\end{figure}

Figure \ref{k-seg} shows the loss vs epochs curve. 

Some sample predictions are shown in table \ref{k-Onlyseg}.


\begin{table}[h!]
\caption{Predicted images from Branch2- standalone - semantic segmentation
(Extreme left - RGB image, Bottom - ground truth, Extreme right - prediction)}
\centering
\def\arraystretch{0.25}% 
\setlength\tabcolsep{10pt}
\begin{tabular}{*{3}{m{0.20\linewidth}}}
\hline
\begin{center}\includegraphics[width=3cm, height=3cm]{k-Onlyseg/000019_10.png}\end{center} &
\begin{center}\includegraphics[width=3cm, height=3cm]{k-Onlyseg/15_gt.png}\end{center} & \begin{center}\includegraphics[width=3cm, height=3cm]{k-Onlyseg/15_pred.png}\end{center}\\
\hline
\begin{center}\includegraphics[width=3cm, height=3cm]{k-Onlyseg/000136_10.png}\end{center} &
\begin{center}\includegraphics[width=3cm, height=3cm]{k-Onlyseg/25_gt.png}\end{center} & \begin{center}\includegraphics[width=3cm, height=3cm]{k-Onlyseg/25_pred.png}\end{center}\\
\hline
\begin{center}\includegraphics[width=3cm, height=3cm]{k-Onlyseg/000083_10.png}\end{center} &
\begin{center}\includegraphics[width=3cm, height=3cm]{k-Onlyseg/35_gt.png}\end{center} & \begin{center}\includegraphics[width=3cm, height=3cm]{k-Onlyseg/35_gt.png}\end{center}\\
\hline
\end{tabular}
\label{k-Onlyseg}
\end{table}

\clearpage

\newpage

\paragraph{Combining Branch1 and Branch2}

This phase involves the training of both the branches such that one branch gives the object detection result and the other branch give segmentation map. The predicted segmentation map is achieved by upsampling the last feature map of the \ac{vgg}16 encoder by 32 times (\ac{fcn}32). Localization, confidence and pixel-wise cross-entropy are calculated in this phase. Images are resized to $300\times300$, and random augmentation is applied on the fly.

\par

Training both the branches parallelly gives mean AP score of \textbf{35.8\%}for object detection and mean IoU score of \textbf{60.9\%}, mean pixel accuracy score of \textbf{95.8\%} for semantic segmentation. 

\begin{figure}[h!]
    \centering
    \includegraphics[width= \linewidth]{results/k-combined.png}
    \caption{Loss vs epoch}
    \label{k-combined}
\end{figure}

Figure \ref{k-combined} shows the loss vs epochs curve. 

Some sample predictions are shown in table \ref{k-combinedpred}

\begin{table}[h!]
\caption{Predicted images from combined architecture - object detection + semantic segmentation}
% (Extreme left - RGB image, Bottom - ground truth, Extreme right - prediction)}
\centering
\def\arraystretch{0.25}% 
\setlength\tabcolsep{10pt}
\begin{tabular}{*{3}{m{0.20\linewidth}}}
\hline
\begin{center}\includegraphics[width=3cm, height=3cm]{k-combined/1.png}\end{center} &
\begin{center}\includegraphics[width=3cm, height=3cm]{k-combined/2.png}\end{center} & \begin{center}\includegraphics[width=3cm, height=3cm]{k-combined/3.png}\end{center}\\
\hline
\begin{center}\includegraphics[width=3cm, height=3cm]{k-combined/000083_10.png}\end{center} &
\begin{center}\includegraphics[width=3cm, height=3cm]{k-combined/20_gt.png}\end{center} & \begin{center}\includegraphics[width=3cm, height=3cm]{k-combined/20_pred.png}\end{center}\\
\begin{center}\includegraphics[width=3cm, height=3cm]{k-combined/000019_10.png}\end{center} &
\begin{center}\includegraphics[width=3cm, height=3cm]{k-combined/25_gt.png}\end{center} & \begin{center}\includegraphics[width=3cm, height=3cm]{k-combined/25_pred.png}\end{center}\\
\begin{center}\includegraphics[width=3cm, height=3cm]{k-combined/000180_10.png}\end{center} &
\begin{center}\includegraphics[width=3cm, height=3cm]{k-combined/0_gt.png}\end{center} & \begin{center}\includegraphics[width=3cm, height=3cm]{k-combined/0_pred.png}\end{center}\\
\hline
\end{tabular}
\label{k-combinedpred}
\end{table}




\paragraph{Combining Branch1 and Branch2 using Lambda in loss function - (FCN32)}

\par The predicted segmentation map is achieved by upsampling the last feature map of the \ac{vgg}16 encoder by 32 times (\ac{fcn}32). Images are resized to $300 \times 300$, and random augmentation is applied on the fly.

\begin{itemize}
    \item Case1: $\lambda_{1}=1$ and $\lambda_{2}$ = 0.01
    
\begin{equation}
    L = \lambda_{1} (-\Sigma_{classes}\;y_{true}\;log(y_{pred})) + \lambda_{2}(\frac{1}{N} (L_{conf} (x,c) + \alpha L_{loc} (x,l,g))) 
\end{equation}

Here, the object detection part will act as a regularizer for the semantic segmentation part. This setup of $\lambda$ value gives a mean IoU score of \textbf{61.56\%} and mean pixel-accuracy score of \textbf{95.7\%} for semantic segmentation. The mAP score for object detection drops down to only \textbf{0.6\%}.

Figure \ref{k-seg-lambda} shows the loss vs epoch curve

\begin{figure}[h!]
    \centering
    \includegraphics[width=\linewidth]{results/k-seg-labda.png}
    \caption{Segmentation loss vs epochs}
    \label{k-seg-lambda}
\end{figure}

\newpage
\item Case2: $\lambda_{1}=0.01$ and $\lambda_{2}$ = 1

\begin{equation}
    L = \lambda_{1} (-\Sigma_{classes}\;y_{true}\;log(y_{pred})) + \lambda_{2}(\frac{1}{N} (L_{conf} (x,c) + \alpha L_{loc} (x,l,g))) 
\end{equation}

As per the loss function above, the semantic segmentation part will act as a regularizer for the object detection part.

\begin{figure}[h!]
\begin{multicols}{2}
    \includegraphics[width=\linewidth]{results/k-confidence.png}\par \includegraphics[width=\linewidth]{results/k-localization.png}\par
    \end{multicols}
\caption{Confidence and localization loss vs epochs}
\label{k_conf_loc}
\end{figure}

\begin{figure}[h!]
\centering
    \includegraphics[width= \linewidth]{results/k-val-lambda.png}
    \caption{Total loss vs epochs}
    \label{k_totallosslambda}
\end{figure}

Figures \ref{k_conf_loc} and \ref{k_totallosslambda} are confidence, localization and total loss curves. 
\par
The mean AP score for object detection is \textbf{35.21\%} while the mean IoU and mean pixel-accuracy scores are \textbf{46.91\%} and \textbf{93.8\%}.

\end{itemize}


\paragraph{Combining Branch1 and Branch2 - (FCN32)}

This experiment is carried out by resizing the images to $512\times512$ and passing it to the network. Augmentation of images are done on the fly. The predicted segmentation map is achieved by upsampling the last feature map of the \ac{vgg}16 encoder by 32 times (\ac{fcn}32). Augmentation of images are done on the fly.

Training both the branches gives mean AP score of \textbf{42.4\%} for object detection while mean IoU and mean pixel-accuracy scores are \textbf{62.60\%} and \textbf{95.7\%} for semantic segmentation. 

\begin{figure}[h!]
\centering
    \includegraphics[width= \linewidth]{results/k-512.png}
    \caption{Total loss vs epochs}
    \label{k_fcn512}
\end{figure}

Figure \ref{k_fcn512} shows the loss vs epoch curve. 


\paragraph{Combining Branch1 and Branch2 - (FCN8)}

This experiment is carried out by resizing the images to $300\times300$ and passing it to the network. Augmentation of images are done on the fly. The predicted segmented map is achieved by upsampling and using skip connections as mentioned in section \ref{SS}

Training both the branches parallelly gives mean AP score of \textbf{0.6\%}for object detection while mean IoU and mean pixel-accuracy of \textbf{50.6\%} and  \textbf{93.2\%}for semantic segmentation.

\begin{figure}[h!]
\centering
    \includegraphics[width= \linewidth]{results/k-fcn8.png}
    \caption{Total loss vs epochs}
    \label{k_fcn8}
\end{figure}

Figure \ref{k_fcn8} shows the loss vs epoch curve. 

\clearpage

\paragraph{Summary of results for KITTI}

\begin{table}[h!]
\begin{tabular}{|c|c|c|c|}
\hline
\multicolumn{4}{|c|}{\textbf{Standalone}}                                     \\ \hline
                                & \textbf{mAP} & \textbf{mIoU} & \textbf{mPA} \\ \hline
Branch1 (object detection) (300 x 300)      & 35.95\%      & -             & -            \\ \hline
Branch2 (semantic segmentation) (300 x 300) & -            & 60.83\%       & 95.6\%       \\ \hline
\multicolumn{4}{|c|}{\textbf{Combined}}                                       \\ \hline
                                & \textbf{mAP} & \textbf{mIoU} & \textbf{mPA} \\ \hline
Combined (FCN32) (300 x 300)    & 35.8\%       & 60.9\%        & 95.8\%       \\ \hline
\begin{tabular}[c]{@{}c@{}}Combined (($\lambda$=1)segmentation loss)+\\ ($\lambda$=0.01)detection loss (FCN32) (300 x 300)\end{tabular} & 0.6\%   & 61.56\% & 95.7\% \\ \hline
\begin{tabular}[c]{@{}c@{}}Combined (($\lambda$=0.01)segmentation loss)+\\ ($\lambda$=1)detection loss (FCN32) (300 x 300)\end{tabular} & 35.21\% & 46.91\% & 93.8\% \\ \hline
Combined (FCN8) (300 x 300)     & 0.6\%        & 50.6\%        & 93.2\%       \\ \hline
Combined(FCN32) (512x512)       & \textbf{42.4\%}    & \textbf{62.60\%}     & \textbf{95.7\%}    \\ \hline
\end{tabular}
\end{table}

\clearpage
\afterpage{\null\newpage}


\rhead{\textit{Discussion and summary}}
\lhead{\thepage}

\chapter{Discussion and summary}

We propose an approach to join object detection and semantic segmentation architecture using a common encoder in this work. \cite{liu2016ssd} is used for object detection in our work. It is a one-stage detector that skips the region proposal step and runs detection directly. It consists of six auxiliary layers of different scales that help it detecting objects of different size. All of these layers are connected to two different heads. One of the head does the classification part, and the other head does the regression part. Also, anchors are overlayed at every grid of the feature map or the image. These anchors provide a starting point for the regressor to predict bounding boxes.
\par
For semantic segmentation we use \cite{long2015fully}. This architecture is based on fully convolutional networks that consist of only convolutional layer and max-pooling layers. It is based on an encoder-decoder architecture. The encoder does the feature extraction part of the images by reducing the spatial resolution of the image, and the decoder does the upsampling of the feature map to regain the same size as the image.
\par
Both the above-mentioned architectures use an encoder to extract image features, and these features are used for object detection as well as semantic segmentation. Using this encoder, we combined both the architectures such that the last feature map is shared between both the architectures. 
A pre-trained \ac{vgg}16 encoder is used to extract features for the combined architecture. 
\par
In order to train the combined architecture, the losses of object detection and semantic segmentation have to be combined. Hence, localization and confidence loss of object detection is combined with the pixel-wise cross-entropy loss of semantic segmentation. 
\par
Several experiments are done to understand the flaws or benefits of using a combined architecture as compared to stand-alone architecture. 
\clearpage
\pagestyle{plain}
\afterpage{\null\newpage}


%%%%%%%%%%%%%%%%%%%%%%%%%%%%%%%%%%%%%%%%%%%%%%%%%%%%%%%%%%%%
%% Literaturverzeichnis
%%
%% Einbinden des Literaturverzeichnisses. Das Makro erwartet
%% ein BibTex-Files (z.B. literatur.bib) ohne Dateiendung
%% als Parameter.
%%%%%%%%%%%%%%%%%%%%%%%%%%%%%%%%%%%%%%%%%%%%%%%%%%%%%%%%%%%%
\bibliographystyle{template/bst/rrlab_english}
%\bibliographystyle{unsrtnat} % Use for unsorted references  
%\bibliographystyle{plainnat} % use this to have URLs listed in References

\bibliography{literatur}




%%%%%%%%%%%%%%%%%%%%%%%%%%%%%%%%%%%%%%%%%%%%%%%%%%%%%%%%%%%%
%% index generieren? ggf. auskommentieren
%%%%%%%%%%%%%%%%%%%%%%%%%%%%%%%%%%%%%%%%%%%%%%%%%%%%%%%%%%%%
% \RRLABindex

%%%%%%%%%%%%%%%%%%%%%%%%%%%%%%%%%%%%%%%%%%%%%%%%%%%%%%%%%%%%
%% Ende des Dokuments
%%%%%%%%%%%%%%%%%%%%%%%%%%%%%%%%%%%%%%%%%%%%%%%%%%%%%%%%%%%%
\end{document}
